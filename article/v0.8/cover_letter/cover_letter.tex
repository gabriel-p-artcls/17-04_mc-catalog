\documentclass{article}
\usepackage{graphicx}
\usepackage{hyperref}
\usepackage{natbib,twoopt}
\usepackage{microtype}
\usepackage{afterpage}
\usepackage{amsmath}

\usepackage{xcolor}
\newcommand{\finish}{\textcolor{red}{\textbf{$\;\;$|-- FINISH --|$\;\;$}}}

\begin{document}

\title{Corrections suggested by the Editor}
\author{GI Perren, AE Piatti, RA V\'azquez}

\maketitle


\begin{abstract}
We thank Dr Tolstoy for her proposed corrections to our article.

Below we address these corrections, plus one extra modification we would like to
incorporate into the article, described in Sect.~\ref{sec:C}.
%
If these changes are accepted by Dr Tolstoy (or by the referee, if the article
is sent back to them), we will immediately send the tex and bbl files for
language editing.

The compiled PDF with these modifications is attached along with this letter.
For convenience, we generated it without images.
\end{abstract}

\clearpage

\renewcommand\thesubsection{\Alph{subsection}}
\subsection{Footnotes}
The footnotes have been incorporated into the text, as suggested.


%
\subsection{Proposal IDs}
As the data used in this work has already been analyzed and published in
previous articles (see Table 1), all the relevant information regarding the
facilities can be found in the original publications.
We prefer to avoid repeating that here, unless Dr Tolstoy feels it is necessary.
In that case, we could add the following to the Acknowledgments section:\\

\noindent \emph{This work is based on observations made at Cerro Tololo
Inter-American Observatory (CTIO), which is operated by the Association of
Universities for Research in Astronomy (AURA), Inc., under cooperative agreement
with the National Science Foundation (NSF).
%
This work is based on observations made at La Silla European Southern
Observatory (ESO).
%
This research draws upon data as distributed by the NOAO Science Archive.
NOAO is operated by AURA, under a cooperative agreement with the NSF.}





%
\subsection{Crowding in the central regions of massive clusters}
\label{sec:C}

In our previous letter we wanted to respond to the referee's comments as quick
as possible, and we feel we rushed the response.
We present here the explanation for our proposed modification to the article,
marked in boldface (Sect. 5.2.1, and Appendix A) in the version attached along
with this letter.

We apologize for the inconveniences this may cause.\\

In the cover letter for Revision 2, the referee correctly pointed out that the
most likely reason for the mass underestimation by \texttt{ASteCA} was the
crowding effect. We agreed, responding\\


\noindent ``\emph{Although \texttt{ASteCA} considers how other photometric effects
alter the generated synthetic CMDs (limiting magnitude, faint stars
incompleteness, photometric errors), it does not include the loss of stars in
the central regions of a massive cluster due to photometric crowding.
(...) This percentage of lost stars in the cluster region couldn't be
accounted for by any of the photometric process included in \texttt{ASteCA}.}''\\

We introduced this effect as the responsible for the mass differences in Sect.
5.2.1 of the article, and closed that section stating\\

\noindent ``\emph{A modelization of this process on our synthetic CMDs is
planned for future releases of the code.}''\\

The above statements are somewhat misleading, since \texttt{ASteCA} does in fact
account for the loss of the \emph{faintest} stars in the synthetic CMDs.

What the code can not do -- as it works with already processed photometric
tables, not with raw fits images -- is a proper artificial stars
test~\citep[as described e.g. in][]{Aparicio_Gallart_1995}. This
is the only way to correctly asses the loss of sources due to crowding (and
blending), and its dependence with observed magnitudes. Instead, \texttt{ASteCA}
approximates the loss of faint stars using the observed luminosity
function~\citep[mentioned in][]{Perren_2015}.
Briefly, this process involves the following steps:

\begin{enumerate}
    \item Obtain the LF of the observed region.
    \item Identify the magnitude value where the maximum count of stars occurs,
    $mag_{max}$. For smaller magnitudes (brighter stars) the data is
    assumed to be 100\% complete.
    \item Model the drop in stars count from $mag_{max}$ towards fainter
    magnitudes.
    \item Remove from each generated synthetic CMD a proportional number of
    stars beyond $mag_{max}$, given by the model obtained in the above step.
\end{enumerate}

This algorithm makes the assumptions that: a) the loss of stars can be roughly
estimated from the region's LF, and b) only the faintest stars will be
affected by incompleteness. For clusters with low densities (open
clusters), this approximation works rather well, but it breaks down as the
observed clusters grow in mass (globular clusters). The reason is that massive
clusters are heavily affected by crowding, whereas sparse clusters are not.
%
As demonstrated in~\cite{Mateo_1988}, crowding in globular clusters affects the
completeness of stars in the entire observed magnitude range, not just at their
faintest end. This is particularly true in the densest central regions of such
clusters.\\

We thus slightly modified Sect 5.2.1 of the article to clearly state that the 
(approximated) completeness correction performed by \texttt{ASteCA} will be
\emph{extended}, to accept a user defined completeness function. The latter is
assumed to be obtained through a proper artificial stars test on the observed
frames.
A small modification was also introduced in Appendix A, regarding the effect of
crowding in the synthetic MASSCLEAN validation clusters.

% Out of the approximately 7 magnitudes spanned on average by the CMDs of our
% 239 observed clusters, only the largest 2.5 magnitudes (at best)
% are corrected by completeness.


\subsection{Minimal changes}
We added the anonymous referee to the Acknowledgments section.
%-------------------------------------------------------------------
\bibliographystyle{aa}
\bibliography{biblio} % your references Yourfile.bib

\end{document}
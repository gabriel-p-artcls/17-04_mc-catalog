% mnras_template.tex
%
% LaTeX template for creating an MNRAS paper
%
% v3.0 released 14 May 2015
% (version numbers match those of mnras.cls)
%
% Copyright (C) Royal Astronomical Society 2015
% Authors:
% Keith T. Smith (Royal Astronomical Society)

% Change log
%
% v3.0 May 2015
%    Renamed to match the new package name
%    Version number matches mnras.cls
%    A few minor tweaks to wording
% v1.0 September 2013
%    Beta testing only - never publicly released
%    First version: a simple (ish) template for creating an MNRAS paper

%%%%%%%%%%%%%%%%%%%%%%%%%%%%%%%%%%%%%%%%%%%%%%%%%%
% Basic setup. Most papers should leave these options alone.
\documentclass[a4paper,fleqn,usenatbib]{mnras}

% Use placeholders instead of loading images.
% \documentclass[a4paper,fleqn,usenatbib,draft]{mnras}
% Don't use a4paper as suggested here:
% https://www.mrao.cam.ac.uk/~dag/my-mnras-advice.html
% \documentclass[fleqn,usenatbib,draft]{mnras}

% New commands that marks parts that need work.
\newcommand{\finish}{\textcolor{red}{\textbf{$\;\;$<-- FINISH -->$\;\;$}}}
\newcommand{\cmmt}[1]{\textcolor{red}{\textbf{$\;\;$<-- #1 -->$\;\;$}}}

% Hold *all* images, no placeholders.
% \usepackage{comment}
% \excludecomment{figure}
% \let\endfigure\relax
% \expandafter\let\csname figure*\endcsname\figure
% \expandafter\let\csname endfigure*\endcsname\endfigure

% MNRAS is set in Times font. If you don't have this installed (most LaTeX
% installations will be fine) or prefer the old Computer Modern fonts, comment
% out the following line
\usepackage{newtxtext,newtxmath}
% Depending on your LaTeX fonts installation, you might get better results with one of these:
%\usepackage{mathptmx}
% \usepackage{txfonts}

% Use vector fonts, so it zooms properly in on-screen viewing software
% Don't change these lines unless you know what you are doing
\usepackage[T1]{fontenc}
\usepackage{ae,aecompl}


%%%%% AUTHORS - PLACE YOUR OWN PACKAGES HERE %%%%%

% Only include extra packages if you really need them. Common packages are:
\usepackage{graphicx}	% Including figure files
\usepackage{amsmath}	% Advanced maths commands
\usepackage{amssymb}	% Extra maths symbols

%%%%%%%%%%%%%%%%%%%%%%%%%%%%%%%%%%%%%%%%%%%%%%%%%%

%%%%% AUTHORS - PLACE YOUR OWN COMMANDS HERE %%%%%

% Please keep new commands to a minimum, and use \newcommand not \def to avoid
% overwriting existing commands. Example:
%\newcommand{\pcm}{\,cm$^{-2}$}	% per cm-squared

%%%%%%%%%%%%%%%%%%%%%%%%%%%%%%%%%%%%%%%%%%%%%%%%%%

%%%%%%%%%%%%%%%%%%% TITLE PAGE %%%%%%%%%%%%%%%%%%%

% Title of the paper, and the short title which is used in the headers.
% Keep the title short and informative.
\title[Analysis of Magellanic clusters with \texttt{ASteCA}]{Homogeneous
analysis of 239 Magellanic clusters with \texttt{ASteCA}}

% The list of authors, and the short list which is used in the headers.
% If you need two or more lines of authors, add an extra line using \newauthor
\author[Perren et al.]{
G. I. Perren,$^{1,3}$\thanks{E-mail: gabrielperren@gmail.com}
A. E. Piatti,$^{2,3}$
and R. A. V\'azquez$^{1,3}$
\\
% List of institutions
$^{1}$Facultad de Ciencias Astron\'omicas y Geof\'{\i}sicas (UNLP),
IALP-CONICET, La Plata, Argentina\\
$^{2}$Observatorio Astron\'omico, Universidad Nacional de C\'ordoba, C\'ordoba,
Argentina\\
$^{3}$Consejo Nacional de Investigaciones Cient\'{\i}ficas y T\'ecnicas
(CONICET), Buenos Aires, Argentina
}

% These dates will be filled out by the publisher
\date{Accepted XXX. Received YYY; in original form ZZZ}

% Enter the current year, for the copyright statements etc.
\pubyear{2015}

% Don't change these lines
\begin{document}
\label{firstpage}
\pagerange{\pageref{firstpage}--\pageref{lastpage}}
\maketitle

% Abstract of the paper
\begin{abstract}
We present a catalogue of 239 resolved Magellanic Clouds clusters observed in
the Washington photometric system.
The entire cluster sample was processed with the recently introduced
\texttt{ASteCA} package, which ensures both an automatized and fully
reproducible treatment, together with a statistically based obtention of their
fundamental parameters and uncertainties.
%
The main parameters determined for each cluster with this tool are: metallicity,
age extinction, true distance modulus, and total mass.
%
% Structural parameters such as deprojected distance to the centre of the galaxy,
% radial profile radius (core and tidal are calculated when possible), approximate
% number of members, membership probabilities for all stars within the cluster
% region, and degree of field star contamination are also reported.
%
Our results allow us to generate a truly objective and homogeneous catalogue
composed of structural properties for each cluster, along with a complete
determination of their fundamental parameters.
%
A detailed internal error analysis and a thorough comparison with values taken
from twenty-six published articles was performed. This lead us to conclude that
\texttt{ASteCA} can be applied to the unsupervised determination of fundamental
cluster parameters; a task of increasing importance as more data becomes
available through upcoming surveys.
%
We present results for the distribution of all fundamental parameters in both
Clouds, along with the obtention of the age-metallicity relation from
our homogeneous set of star cluster's ages and metallicities.
\end{abstract}

% Select between one and six entries from the list of approved keywords.
% Don't make up new ones.
\begin{keywords}
galaxies: star clusters: general -- Magellanic Clouds --
techniques: photometric -- methods: statistical
\end{keywords}



%%%%%%%%%%%%%%%%%%%%%%%%%%%%%%%%%%%%%%%%%%%%%%%%%%%%%%%%%%%%%%%%%%%%%%%%%%%%%%%%
%%%%%%%%%%%%%%%%% BODY OF PAPER %%%%%%%%%%%%%%%%%%%%%%%%%%%%%%%%%%%%%%%%%%%%%%%%

\section{Introduction}
\label{sec:intro}

The study of a galaxy's structure, dynamics, star formation history, chemical
enrichment history, etc., depends heavily on the analysis of its star clusters.
These galactic building blocks are made up of a varying number of coeval stars
which share a chemical composition, are located at the same distance, and
affected by roughly the same amount of reddening. All these factors greatly
facilitate the obtention of their fundamental parameters, and thus the
properties of the host galaxy.
%
New developments in astrophysical software allow the homogeneous
processing of different types of star clusters' databases. The article series
by Kharchenko et al. 
\citep[see][and references therein]{Kharchenko_2005,Schmeja_2014}
and the integrated photometry based derivation of age and mass for 920 clusters
presented
in~\cite{Popescu_2012}\footnote{Based on their \texttt{MASSCLEAN} package:
\url{http://www.massclean.org/}}, are examples of semi-automated and
automated packages applied on a large number of clusters.

Such tools notwithstanding, there is no guarantee that employing a homogeneous
method will result in similar -- much less unique -- parameter values
across different studies. This is particularly true when the methods require
user intervention, which inevitably makes the results subjective to some degree.
%
In~\cite{Netopil_2015} the parameters age, extinction, and distance are
contrasted throughout seven published databases, for open clusters matched
across them (metallicity is not inspected as this quantity is usually assumed
rather than adjusted).
The authors find that all articles show non-negligible offsets among each
other, in the reported parameter values for the studied star clusters.
%
This result underscores an important issue.
Most of the times CMD isochrone fits are done by-eye, adjusting correlated
parameters independently and often omitting any kind of proper error treatment
\citep[see][for a more detailed description of this problem]{vonHippel_2014}.
When statistical methods are employed, the code used is seldom publicly shared
to allow scrutiny by the community. There is then no objective way to asses the
underlying reliability of each set of results.
Lacking this basic audit, the decision of which database values to use
becomes a matter of preference.

% Furthermore,
As demonstrated by~\cite{Hills_2015}, assigning precise fundamental
parameters for an observed cluster (OC) is not a straightforward task. Using
multiple combinations of up to eight filters ($UBVRIJHK_s$) and three stellar
evolution models, they process an open cluster with their Bayesian isochrone
fitting technique\footnote{BASE-9: 
\url{http://webfac.db.erau.edu/~vonhippt/base9/}}, and arrive at statistically
different results for its final parameter values.
%
The above study was performed on NGC 188, a ${\sim}4$ Gyr Milky Way OC with a
well defined sequence that is rather unaffected by field star contamination, and
with proper motions and radial velocities data available.
One can therefore expect that analysing clusters with a complex or non-standard
morphology, affected by a significant amount of field star contamination,
observed with fewer filters, and without information about its dynamics (i.e.:
a rather standard situation), will be significantly more complicated.
%
A mismatch between several theoretical evolutionary models among themselves,
along with an inability to reproduce OCs in the unevolved main sequence
range, had already been reported in~\cite{Grocholski_2003}.

The aforementioned difficulties in the analysis of an OC will only increase if
the study is done by-eye, since: a) the number of possible solutions manually
fitted is several orders of magnitude smaller than what a code can
handle, b) correlations between parameters are almost entirely disregarded, c)
uncertainties can not be assigned through valid mathematical
means -- and are often not assigned at \mbox{all --}, and d) the final values
are necessarily highly subjective.
%
The need is clear for an automated general method with a fully open and
extensible code base, that takes as much information into account as possible,
and is capable of generating reproducible results.\\

In \cite{Perren_2015} (hereafter Paper I) we presented the Automated Stellar
Cluster Analysis
(\texttt{ASteCA}\footnote{\texttt{ASteCA} is released under a general public
license (GPL v3; \url{https://www.gnu.org/copyleft/gpl.html}) and
can be downloaded from its official site: \url{http://asteca.github.io}})
package, aimed towards allowing a more accurate and comprehensive study of star
clusters.
Through an unassisted process the code analyses an OC's positional and
photometric data to derive its fundamental parameters, along with their
uncertainties.
\texttt{ASteCA} was applied on a group of 20 observed Milky Way open clusters
in Paper I. As shown in that article, the code is able to assign precise
parameter values for OCs with low to medium field star contamination, and gives
reasonable estimations for heavily contaminated OCs.
Every part of this astrophysical package is open and publicly available, and its
development is ongoing.

In the present work we apply \texttt{ASteCA} on 239 OCs from the Small and Large
Magellanic Clouds (S/LMC), distributed up to ${\sim}5^{\circ}$ and ${\sim}8^
{\circ}$ in angular distance from their centres, respectively.
%
The Magellanic Clouds (MCs) are located close enough to us to allow the study of
their resolved star clusters. The large number of catalogued
OCs -- $\sim4000$ are listed in the~\cite{Bica_2008} catalogue -- makes them an
invaluable resource for investigating the properties of the two most massive
galaxies that orbit the Milky Way.
%
The reddening that affects the MCs is relatively small,
except for a few regions like 30 Doradus in the LMC, where $E_{(B-V)}$
extinction can reach values above $0.4$ mag~\citep{Piatti_2015b}.
The overall low levels of extinction further simplifies the research of these
two galaxies, through the obtention of their clusters' parameters.
%
We make use of the $CT_1$ filters of the Washington photometric
system~\citep{Canterna_1976,Geisler_1996}, which are known to produce a colour
that is highly sensitive to metal abundance \citep{Geisler_1999}.
The results obtained here regarding the metal content of the clusters in our
sample, are thus of clear relevance for the analysis of the MCs chemical
enrichment history.

This is the first study where such a large sample of resolved star clusters is
homogeneously analysed in an automatic way, with all of their fundamental
parameters statistically estimated rather than fitted by eye or fixed a priori.
%
Having metal content assigned for 100\% of our sample is particularly
important, especially considering the state of other catalogues. The latest
version of the well known DAML02 database\footnote{Latest version: v3.5, 2016
Jan 28; \url{http://www.wilton.unifei.edu.br/ocdb/}}
\citep{Dias_2002} for example, reports abundances for only 13\% of the 2167
clusters it contains.

In Sect.~\ref{sec:clust-sample} we present the OC sample used in
this work along with the twenty-six studies used to compare and validate our
results.
Sections~\ref{sec:fund-params} and~\ref{sec:errors-fit} describe the obtention
of the fundamental parameters found with \texttt{ASteCA}, and analyse their
uncertainties respectively.
A detailed comparison of our results with published values from the
aforementioned articles is performed in~\ref{sec:comp-pub-data}.
Sect.~\ref{sec:param-dist} shows how the fundamental parameters obtained with
\texttt{ASteCA} for both Clouds are distributed, along with the galaxies'
age-metallicity relations.
We give in Sect.~\ref{sec:summ-concl} a summary of our results and concluding
remarks.


%%%%%%%%%%%%%%%%%%%%%%%%%%%%%%%%%%%%%%%%%%%%%%%%%%%%%%%%%%%%%%%%%%%%%%%%%%%%%%%%
%%%%%%%%%%%%%%%%%%%%%%%%%%%%%%%%%%%%%%%%%%%%%%%%%%%%%%%%%%%%%%%%%%%%%%%%%%%%%%%%

\section{Observed clusters sample}
\label{sec:clust-sample}

The data used in this work consists of $CT_1$ Washington photometric system
magnitudes and their errors, for the 239 OCs that comprise our sample.
The majority of the OCs, 150 out of the 239, belong to the LMC while the
remaining 89 belong to the SMC.
%
In Fig.~\ref{fig:ra-dec} we show the distribution of all OCs in both MCs,
along with the entire set of star clusters from the~\cite{Bica_2008} database
for reference.

\begin{figure}
\includegraphics[width=\columnwidth]{figures/RA_DEC.png}
\caption[Pass this otherwise it chokes with the cite]{Distribution of our set of
analysed OCs, shown in red over the~\cite{Bica_2008} database of 3740
star clusters (black dots) for both MCs. The assumed centres of the clouds are
marked with a green triangle.}
\label{fig:ra-dec}
\end{figure}


All OCs have been previously analysed and the results published in the articles
shown in Table~\ref{tab:literature}.
These nineteen articles can be merged into a single group, whose results
arise from the analysis of the same $CT_1$ photometry used by \texttt{ASteCA} in the
current study. From here on we will refer to this group as the ``literature''
and the values presented in each of the articles in Table~\ref{tab:literature}
as ``literature values''.
Details on the data handling and reduction process can be found in the
literature, and will not be repeated here.
%
Metallicity, age, reddening, and true distance modulus ($\mu_{\circ}$) values
are assigned for all OCs in the literature, except for the 36 clusters presented
in~\cite{Piatti_2011b} which had only their ages estimated via the $\delta T_1$
index \citep{Phelps_1994,Geisler_1997}.
In most works metallicity is fixed to $z=0.004$ and $z=0.008$ ($[Fe/H]=-0.7$,
$[Fe/H]=-0.4$), while the distance modulus values are always fixed to
$\mu_{\circ}{=}18.5$ mag and $\mu_{\circ}{=}18.9$ mag, for the S/LMC
respectively.
Ages and extinctions reported in the literature were in all cases obtained by
eye, either through the standard isochrone technique or applying the
$\delta T_1$ index method.
%
\cite{Maia_2013} is the only article that presents total mass estimations for
their OCs sample.

Our set of 239 OCs was also cross-matched with databases from seven
published articles, as shown in Table~\ref{tab:databases}.
These can be combined into a single group of works, were different
photometry was used to obtain the parameter values of each OC. This group will
be referred from here on as the ``databases'' (DBs), as a way to distinguish
them from the ``literature'' articles mentioned above.
%
In Sect.~\ref{sec:comp-pub-data} we show how the parameter values obtained
by \texttt{ASteCA} for our sample of OCs, compares to those given in both the
literature and the databases.

\begin{table} 
\centering
 \caption{Set of articles where the same photometric data was used as that
 employed by \texttt{ASteCA} in this work.
 $N$ is the number of clusters in this combined ``literature'' database that are
 present in the  corresponding article. The CTIO 0.9m, 1.5m, and Blanco 4m
 telescopes are  located at Cerro Tololo (Chile); the Danish 1.54m telescope is
 located at La Silla Observatory (Chile).}
\label{tab:literature}
 \begin{tabular}{l c c c}
\hline
Article & $N$ & Galaxy & Telescope \\
\hline
%~\cite{Geisler_2003}, G03 & 8 & LMC & CTIO 0.9m \\ 
%~\cite{Piatti_2003_0}, P03 & 6 & LMC & CTIO 0.9m \\ 
%~\cite{Piatti_2003a}, P03a & 5 & LMC & CTIO 0.9m \\ 
%~\cite{Piatti_2005}, P05 & 8 & SMC & CTIO 0.9m \\ 
%~\cite{Piatti_2007_0}, P07 & 4 & SMC & CTIO 0.9m \\ 
%~\cite{Piatti_2007a}, P07a & 2 & SMC & Danish 1.54m \\ 
%~\cite{Piatti_2007b}, P07b & 2 & SMC & Danish 1.54m \\ 
%~\cite{Piatti_2008}, P08 & 6 & SMC & Danish 1.54m \\ 
%~\cite{Piatti_2009}, P09 & 5 & LMC & CTIO 0.9m / Danish 1.54m\\ 
%~\cite{Piatti_2011_0}, P11 & 14 & SMC & CTIO 1.5m \\ 
%~\cite{Piatti_2011a}, P11a & 3 & LMC & CTIO 0.9m \\ 
%~\cite{Piatti_2011b}, P11b & 9 & SMC & Blanco 4m \\ 
%~\cite{Piatti_2011c}, P11c & 36 & LMC & Blanco 4m \\ 
%~\cite{Piatti_2011d}, P11d & 11 & SMC & Blanco 4m \\ 
%~\cite{Piatti_2012a}, P12a & 26 & LMC & Blanco 4m \\ 
%~\cite{Piatti_2012b}, P12b & 4 & SMC & Blanco 4m \\ 
%~\cite{Palma_2013}, P13 & 23 & LMC & Blanco 4m \\ 
%~\cite{Maia_2013}, M13 & 29 & SMC & Blanco 4m \\ 
%~\cite{Choudhury_2015}, C15 & 38 & LMC & Blanco 4m \\ 
\cite{Geisler_2003} & 8 & LMC & CTIO 0.9m \\ 
\cite{Piatti_2003a} & 5 & LMC & CTIO 0.9m \\ 
\cite{Piatti_2003b} & 6 & LMC & CTIO 0.9m \\ 
\cite{Piatti_2005} & 8 & SMC & CTIO 0.9m \\ 
\cite{Piatti_2007a} & 4 & SMC & CTIO 0.9m \\ 
\cite{Piatti_2007b} & 2 & SMC & Danish 1.54m \\ 
\cite{Piatti_2007c} & 2 & SMC & Danish 1.54m \\ 
\cite{Piatti_2008} & 6 & SMC & Danish 1.54m \\ 
\cite{Piatti_2009} & 5 & LMC & \vtop{\hbox{\strut CTIO 0.9m /}
									 \hbox{\strut Danish 1.54m}} \\ 
\cite{Piatti_etal_2011a} & 3 & LMC & CTIO 0.9m \\ 
\cite{Piatti_etal_2011b} & 14 & SMC & CTIO 1.5m \\ 
%
\cite{Piatti_2011a} & 9 & SMC & Blanco 4m \\ 
\cite{Piatti_2011b} & 36 & LMC & Blanco 4m \\ 
\cite{Piatti_2011c} & 11 & SMC & Blanco 4m \\ 
%
\cite{Piatti_2012a} & 26 & LMC & Blanco 4m \\ 
\cite{Piatti_2012b} & 4 & SMC & Blanco 4m \\ 
\cite{Palma_2013} & 23 & LMC & Blanco 4m \\ 
\cite{Maia_2013} & 29 & SMC & Blanco 4m \\ 
\cite{Choudhury_2015} & 38 & LMC & Blanco 4m \\ 
\hline
 \end{tabular} 
\end{table}

\begin{table}
\centering
  \caption{Set of articles where different photometric data was used to analyse
  the OCs sample.
  $N$ is the number of clusters in each database that could be cross-matched
  with our set of 239 clusters. The photometry used in each article is shown in
  the ``Phot'' column.}
\label{tab:databases}
 \begin{tabular}{l c c c}
\hline
Article & $N$ & Galaxy & Phot\\
\hline
\cite{Pietrzynski1999}, P99 & 7 & SMC & $BVI$ \\ 
\cite{Pietrzynski2000}, P00 & 25 & LMC & $BVI$ \\ 
\cite{Hunter_2003}, H03 & 62 & S/LMC & $UBVR$ \\ 
\cite{Rafelski_2005}, R05 & 24 & SMC & $UBVI$ \\ 
\cite{Chiosi_2006}, C06 & 16 & SMC & $VI$ \\ 
\cite{Glatt_2010}, G10 & 61 & S/LMC & $UBVI$ \\ 
\cite{Popescu_2012}, P12 & 48 & LMC & $UBVR$ \\ 
\hline
 \end{tabular} 
\end{table}
  



%%%%%%%%%%%%%%%%%%%%%%%%%%%%%%%%%%%%%%%%%%%%%%%%%%%%%%%%%%%%%%%%%%%%%%%%%%%%%%%%
%%%%%%%%%%%%%%%%%%%%%%%%%%%%%%%%%%%%%%%%%%%%%%%%%%%%%%%%%%%%%%%%%%%%%%%%%%%%%%%%

\section{Obtention of observed clusters parameters}
\label{sec:fund-params}

All the fundamental -- metallicity, age, true distance modulus,
extinction, mass -- and structural parameters -- centre coordinates, radius,
contamination index, approximate number of members, membership probabilities,
true cluster probability -- associated with the OCs in our database, were
obtained either automatically or semi-automatically by the \texttt{ASteCA}
package.
%
A detailed description of the functions built within this tool can be found in
Paper I, and in the code's online documentation\footnote{\texttt{ASteCA}
documentation: \url{http://asteca.rtfd.org}}.
  
The final catalogue, composed of the fundamental and structural parameters for
the 239 resolved Magellanic star clusters in our sample, can be
accessed via Vizier\footnote{\url{http://vizier.XXXX}}.
We've made available online the entire Python codebase developed to analyse
the data obtained with \texttt{ASteCA} and generate the figures in the
article.\footnote{\url{https://github.com/Gabriel-p/mc-catalog}}
%
In addition, the full output image generated by \texttt{ASteCA} for each
processed cluster together with its corresponding membership probability file,
can be accessed separately through a public code
repository.\footnote{\url{https://github.com/Gabriel-p/mc-catalog-figs}}


%%%%%%%%%%%%%%%%%%%%%%%%%%%%%%%%%%%%%%%%%%%%%%%%%%%%%%%%%%%%%%%%%%%%%%%%%%%%%%%%

\subsection{Ranges for fitted fundamental parameters}
\label{ssec:param-ranges}

Before \texttt{ASteCA} is able to process an OC, the user is required to provide
a suitable range for each free (fitted) fundamental parameter. This is
accomplished by setting a minimum, a maximum, and a step value in the
appropriate input data file.
%
As explained in Sect.~\ref{ssec:isoch-fit}, each single value combination
for each of the five fundamental parameters fitted represents a unique synthetic
cluster or model.
%
Hence an interval too broad between the minimum and maximum
values for a given parameter, or a very small step -- or a combination of
both -- will result in a large number of possible values.
The larger the number of values a parameter can access, the larger the number of
models the code will have to process to find the one with the best match to the
OC\@.
%
For this reason, the allowed ranges and steps were selected to provide
a balance between a reasonably large interval, and a computationally manageable
number of total models.
Especial care must be taken to avoid defining a range that could bias the
results towards a particular region of any fitted parameter.
All ranges can be seen in Table~\ref{tab:ga-range}, with the rationale behind
each one explained below.\\

% Metallicity
Unlike most previous works where the metallicity is a pre-defined fixed
value (usually $z=0.008$ for the LMC and z=$0.004$ for the SMC), we do not make
any assumptions on the cluster's metal content. The minimum and maximum [Fe/H]
values selected are [$\sim$-2.2, 0]; with a step of $\sim$0.1 dex. This interval
covers completely the usual metallicities reported for OCs in the MCs.
% Age
The age range is set to 1 Myr--12.6 Gyr with a step of
$\log\mathrm{(age/yr)}=0.05$, encompassing almost the entire allowed range of
the CMD service\footnote{CMD\@: \url{http://stev.oapd.inaf.it/cgi-bin/cmd}}
where the theoretical isochrones were obtained from (see
Sect.~\ref{ssec:isoch-fit}).

% Extinction
The maximum allowed value for the extinction of each cluster was determined
through the MCEV\footnote{Magellanic Clouds Extinction Values (MCEV):
\url{http://dc.zah.uni-heidelberg.de/mcextinct/q/cone/form}} reddening maps
\citep{Haschke_2011}, while the minimum value is always zero.
%
These maps contain a large number of observed areas with assigned $E_{V-I}$
extinction values.
The TOPCAT\footnote{TOPCAT\@: \url{http://www.star.bris.ac.uk/~mbt/topcat/}}
tool
was used to query a region as small as possible around the positions of each our
OCs for MCEV reddening values.
For approximately 85\% of our sample we found several
regions with an associated reddening value, within a box of 0.5 deg centred on
the OC's position.
For the remaining OCs, larger boxes had to be used to find regions with assigned
$E_{V-I}$ values. The two most extreme cases are NGC1997
($\alpha{=}5^h30^m34^s$, $\delta{=}-63^\circ12'12''$, [J2000.0]) and OHSC28
($\alpha{=}5^h55^m35^s$, $\delta{=}-62^\circ20'43''$, [J2000.0]) in the outers
of the LMC, where boxes of 4 deg and 6 deg respectively where needed to find a
region with an assigned reddening value.
%
Once all close-by MCEV regions for an OC are found, the maximum value
MCEV$_{\max}$ among all regions is selected as the extinction range's upper
limit. Three steps are used to ensure that the reddening range is partitioned
similarly for all MCEV$_{\max}$ values: 0.01 for MCEV$_{\max}>0.1$, 0.02 if
$0.05\leq \mathrm{MCEV}_{\max}\leq0.1$, and 0.005 for MCEV$_{\max}<0.05$.
%
The $E_{V-I}$ extinction is converted to $E_{B-V}$
following~\cite{Tammann_2003}: $E_{V-I}{=}1.38\,E_{B-V}$, with an extinction
law of $R_v{=}3.1$ applied throughout the analysis.

% Distance modulus
% de Grijs et al. (2015): SMC (m-M)0 = 18.96 +/- 0.02 mag
% de Grijs et al. (2014): LMC (m-M)o = 18.49 +/- 0.09 mag
The true distance modulus ranges for the S/LMC Clouds were assumed to be
$18.96\pm0.1$ mag and $18.46\pm0.1$ mag respectively; where the mean values
are taken from~\cite{de_Grijs_2015} and~\cite{de_Grijs_2014}.
%
% 18.96 +- 0.1 --> 18.86-19.06 --> depth = 5.7 kpc (SMC)
% 18.49 +- 0.1 --> 18.4-18.6   --> depth = 4.6 kpc (LMC)
The 0.1 mag deviations allowed give depths of $\sim5.7$ kpc and
$\sim4.6$ kpc for the S/LMC\@. This covers entirely the line of sight
depths found in~\cite{Subramanian_2009} for both MCs.\footnote{Line of sight
depth, SMC:\ $4.9\pm1.2$ kpc (bar), $4.23\pm1.48$ kpc (disk); LMC:\
$4.0\pm1.4$ kpc (bar), $3.44\pm1.16$ kpc (disk).}

%%Mass and binarity
For the total cluster mass, the range was set to a minimum of
$10\,M_ {\odot}$ and a maximum of $10000\,M_{\odot}$. This is true for almost
all the OCs in our sample, except for 15 visibly massive clusters for
which the maximum mass limit was increased to $30000\,M_{\odot}$.
The step used to divide the mass range was $200\,M_{\odot}$, which also sets the
minimum possible uncertainty for a cluster's mass estimate.

% Rubele et al. (2011); The SFH of the LMC star cluster NGC1751
% The binary fraction is set to a value of 0.3 for binaries with mass ratios in
% the range between 0.7 and 1.0, which is consistent
% with the prescriptions for binaries commonly used in works devoted
% to recover the field SFH in the Magellanic Clouds (e.g. Holtzman
% et al. 1999; Harris & Zaritsky 2001; Javiel, Santiago & Kerber 2005;
% Noel et al. 2009). Notice that this assumption is also in agreement ¨
% with the few determinations of binary fraction for stellar clusters in
% the LMC (Elson et al. 1998; Hu et al. 2010, both for NGC 1818, a
% stellar cluster younger than ∼100 Myr).
Finally, the binary fraction was fixed to a value of 0.5 -- considered a
reasonable estimate for OCs~\citep{von_Hippel_2005,Sollima_2010} -- to avoid
introducing an extra degree of complexity into the fitting process. Secondary
masses are randomly drawn from a uniform mass ratio distribution of the form
$0.7{\le}q{\le}1$, where $q{=}M_2/M_1$, and $M_1$, $M_2$ are the primary and
secondary masses. This range for the secondary masses was found for example for
the LMC cluster NGC 1818 in~\cite{Elson_1998}, and represents a value commonly
used in analysis regarding the
MCs~\citep[see][and references therein]{Rubele_2011}.

\begin{table}
\centering
\caption{Fundamental parameters' ranges used by \texttt{ASteCA} on our set
of 239 OCs. The approximate the number of values used for each parameter is N,
which gives total of $\sim2.3 {\times}10^7$ possible models.}
\label{tab:ga-range}
\begin{tabular}{lcccc}
\hline
\hline\\[-1.85ex]
 Parameter & Min & Max & Step & N\\
\hline\\[-1.85ex]
[Fe/H] & $\sim$-2.2 & 0 & $\sim$0.1 & 23\\
$\log\mathrm{(age/yr)}$ & 6. & 10.1 & 0.05 & 82\\
$E_{B-V}$ & 0.0 & MCEV$_{\max}$ & [0.5, 1, 2]${\times}10^{-2}$ & $\sim$12\\
$\mu_{SMC}$ & 18.86 & 19.06 & 0.02 & 10\\
$\mu_{LMC}$ & 18.4 & 18.6 & 0.02 & 10\\
Mass ($M_{\odot}$) & 10 & [1, 3]${\times}10^{3}$ & 200 & 50/150\\
\hline
\end{tabular}
\end{table}


%%%%%%%%%%%%%%%%%%%%%%%%%%%%%%%%%%%%%%%%%%%%%%%%%%%%%%%%%%%%%%%%%%%%%%%%%%%%%%%%

\subsection{Centre and radius assignment}
\label{ssec:centre-radius}

In fully automatic mode, \texttt{ASteCA} uses a Gaussian kernel density
estimator (KDE) function to determine the centre of the OC as the point of
maximum star density.
A radial density profile (RDP) is used in this same mode to estimate
the OC's radius, as the point where the RDP reaches the star density of the
surrounding field.

The first function requires that the density of cluster members make the
OC stand out over the combination of foreground and background field stars, and
that no other over-density is present in the observed frame.\footnote{The single
over-density limitation is planned to be lifted in upcoming versions of
\texttt{ASteCA}.}
The second function will give good results when the RDP generated is reasonably
smooth, and only if the OC does not occupy the entire observed frame; i.e.:\
some portion of the surrounding field must be visible.
%
When these conditions are not met for either function, the user needs to run
the code on semi-automatic mode. In this mode the centre coordinates can be
either manually fixed, or found by the code based on an initial set of
approximate values. Similarly, in semi-automatic mode the radius must be fixed
by the user.

% Clusters with manual center values: 82, 34%
% Clusters with manual radii values: 97, 40%
For over $\sim66\%$ of our OCs sample the centre coordinates were obtained
via a KDE analysis, based on initial approximate values passed to
the code.
Radii values were calculated by \texttt{ASteCA} for $\sim60\%$ of the OCs,
through an RDP analysis of the observed surrounding field.
%
The remaining OCs are those that are highly contaminated, contain very
few members, and/or occupy most of the observed frame. This means that an
automatic obtention of their centres and/or radii is not possible, and the
semi-automatic mode must be used.

The contamination index ($CI$) calculated by the code, is a parameter related to
the number of foreground/background stars present in the defined cluster region.
A value of $CI{=}0$ means that the OC is not contaminated by field stars,
$CI{=}0.5$ means the overdensity is indistinguishable from the surrounding
field (i.e.:\ the average number of field stars expected within the cluster
region equals the total number of stars in it), and $CI{>}0.5$ means more field
stars are expected within the cluster region than true cluster
members.\footnote{See Paper I, Sect. 2.3.2 for a complete mathematical
definition of the index.}
%
For reference, the average $CI$ of the set of OCs with manual radii assignment
is $CI{\simeq}0.9\pm0.2$, while the same average is $CI{\simeq}0.6\pm0.2$ for
those OCs where the radius was automatically estimated. Naturally, more
contaminated OCs are the ones that require more often that the radius be set
manually, to prevent the code from employing a wrong value in its analysis.


%%%%%%%%%%%%%%%%%%%%%%%%%%%%%%%%%%%%%%%%%%%%%%%%%%%%%%%%%%%%%%%%%%%%%%%%%%%%%%%%

\subsection{Field-star decontamination}
\label{ssec:dencontamination}

A decontamination algorithm (DA) was employed on the $CT_1$
colour-magnitude diagram (CMD) of each processed OC.\@
The DA allows cleaning the CMD of field-star contamination, before the isochrone
fitting function (see~\ref{ssec:isoch-fit}) is used. The DA must therefore
process all stars within the defined cluster region, i.e: the circular region of
centre and radius either automatically determined by the code or manually fixed
by the user.

The Bayesian DA presented in Paper I was improved for this article, the new DA
works in two steps. First, the original Bayesian membership probability (MP)
assignation is applied to the CMD of all stars within the cluster region (see
Paper I for more details).\@
%
After that, a cleaning algorithm is used to remove stars of low MPs from the
CMD, as shown in Fig.~\ref{fig:DA_BF}.
To do this, the full CMD is divided into smaller cells, according to a
given binning method. By default \texttt{ASteCA} uses the Bayesian blocks
method\footnote{\url{http://www.astroml.org/examples/algorithms/plot_bayesian_blocks.html}}
introduced in~\cite{Scargle_2013}, via the implementation of the astroML
package~\citep{Vanderplas_2012}.\footnote{While Bayesian blocks binning is the
default setting in \texttt{ASteCA}, several others techniques for CMD star
removal are available, as well as five more binning methods.}
This second step is similar to the field-star density based cell-by-cell removal
algorithm applied in~\citet[][B07]{Bonatto_2007}, which uses a simpler
rectangular grid.
%
The main difference, aside from the binning method employed, is that
\texttt{ASteCA} can remove stars cell-by-cell based on their previously assigned
MPs, not randomly as done in B07.

For over $\sim$70\% of our sample the default settings in \texttt{ASteCA} were
used in the decontamination process, this is: Bayesian DA followed by the
removal of low MP stars, based on a Bayesian block binning of the cluster
region's CMD.\@
%
The remaining OCs were processed with modified settings to allow a proper
field-star decontamination, mainly for those OCs with a low number of members or
a heavily contaminated evolutionary sequence.
Changes in the settings can go from selecting a different binning
method~\citep[often a rectangular grid using Scott's rule,][]{Scott_1979}, to
skipping the Bayesian MP assignation and only performing a density based
cell-by-cell statistical field star removal (in this last case, the DA works
very similarly to the B07 algorithm).

A proper field-star decontamination is of the utmost importance, as the
resulting -- hopefully -- clean cluster sequence will determine the OCs'
fundamental parameters via the isochrone fitting process that follows.

\begin{figure}
\includegraphics[width=\columnwidth]{figures/L62_DA_BF.png}
\caption{CMDs for the SMC-L62 cluster. \emph{Top}: CMD of cluster region to the
left, where $n_{memb}$ is the approximate number of true members based on the
star density of the OC compared to the star density of the field. The CMD of the
surrounding field region is shown to the right. In both cases $N_{accpt}$ is
the number of stars that were not rejected due to their large photometric
errors. Rejected stars with large errors are shown as green crosses.
\emph{Bottom}: cluster region after the DA is applied is shown at the left; MPs
vary according to the colorbar at the top right.
Dotted horizontal and vertical lines show the binning used to reject low MP
stars cell-by-cell, as obtained via the Bayesian blocks
method. Stars that are drawn semi-transparent in each cell are those that were
rejected, i.e.: not used in the isochrone fitting process that follows.
$N_{fit}$ is the number of stars left after the cell-by-cell rejection.
The best fit isochrone is shown in green and red in this CMD, and in the CMD of
the best match synthetic cluster to the right (generated from that isochrone),
respectively. $N_{synth}$ is the number of stars in the synthetic cluster, and
the dotted lines represent the binning obtained using Knuth's rule.}
\label{fig:DA_BF}
\end{figure}



%%%%%%%%%%%%%%%%%%%%%%%%%%%%%%%%%%%%%%%%%%%%%%%%%%%%%%%%%%%%%%%%%%%%%%%%%%%%%%%%

\subsection{Isochrone fitting}
\label{ssec:isoch-fit}

The last part of the automatic analysis performed by \texttt{ASteCA} is the isochrone
fitting of the OC, to estimate its fundamental parameters.
In rigour, the process does not employ isochrones but rather synthetic clusters
(SC) generated from theoretical isochrones.\footnote{For a detailed description
on the \texttt{ASteCA} technique for the generation of synthetic clusters from
theoretical isochrones see: Paper I, Sect. 2.9.1}
%
In this work we use a PARSEC v1.1 set of isochrones \citep[][B12]{Bressan_2012}.
%
An isochrone of given metallicity and age is combined with a total mass value
and with the log-normal form of the initial mass function (IMF) defined
by~\cite{Chabrier_2001}, to generate a SC with a correct mass distribution.
The IMF is stochastically sampled until the desired total cluster
mass is reached.
%
The position of this SC in the CMD is then affected by reddening and moved a
certain distance modulus value, to obtain the final SC -- or model -- that will
be compared to the OC.\@

The Poisson likelihood rate (PLR) developed in~\cite{Dolphin_2002} is employed
to asses the goodness of match between the OC and a SC.\@ As the PLR is a
binned statistic, it requires binning the CMDs of the OC and all the SCs that
can be generated according to the fundamental parameter ranges defined in Sect.
\ref{ssec:param-ranges}. This part of the analysis uses Knuth's
rule~\citep[][also implemented via the astroML package]{Knuth_2006} as the
default binning method, see bottom right plot in Fig.~\ref{fig:DA_BF}.
The aforementioned Bayesian blocks method results in noticeably larger cells,
which means that if applied here some defining features of the evolutionary
sequence could be lost.
%
The inverted logarithmic form of the PLR can be written as

\begin{equation}
LPLR  = -\ln PLR = \sum_i m_i - n_i + n_i \ln \frac{n_i}{m_i},
\label{eq:likelihood}
\end{equation}

\noindent where $m_i$ and $n_i$ are the number of stars in the $i$th CMD cell of
the SC and the OC, respectively.\footnote{If for any given cell we have
$n_i\neq0$ and $m_i=0$, a very small number is used instead ($m_i=1{\times}10^
{-10}$) to avoid a mathematical inconsistency with the factor $\ln m_i$.}

The best match for an OC is given by the SC that minimizes the LPLR.\@
Since we have five free fundamental parameters, a 5-dimensional surface of
solutions is determined by all the possible SCs that can be matched to each
OC.\@
%
\texttt{ASteCA} applies a genetic algorithm (GA) on this surface to find the SC
that best matches the OC.\@ This SC thus determines the fundamental parameter
values that will be assigned to the OC under analysis.\@

The GA uses a set of reasonable default options which indicate how it performs
the search for the best OC-SC match.
As with the DA, these options had to be modified -- mainly extending the depth
of the search for the best match -- for $\sim$30\% of the OCs in the
sample. This sub-sample is composed of clusters
with particularly complicated morphologies, i.e.:\ very low number of members,
high field star contamination, little to no MS present above the
maximum magnitude value observed, etc. 

Once the GA returns the optimal fundamental parameter values found for an OC,
uncertainties are estimated via a standard bootstrap technique. This process
takes a significant amount of time to complete, since it involves running the GA
several more times on a randomly generated OC with replacement.\footnote
{Generating a new OC ``with replacement'', means randomly picking stars one by
one from the original OC, where any star can be selected more than once. The
process stops when the same number of stars as those present in the original OC
have been picked.} Ideally, the bootstrap process would be run hundreds and even
thousands of times. This is not possible in practice, as it would be
prohibitively costly timewise, so we must settle with running it ten times for
each OC in our set.




%%%%%%%%%%%%%%%%%%%%%%%%%%%%%%%%%%%%%%%%%%%%%%%%%%%%%%%%%%%%%%%%%%%%%%%%%%%%%%%%
%%%%%%%%%%%%%%%%%%%%%%%%%%%%%%%%%%%%%%%%%%%%%%%%%%%%%%%%%%%%%%%%%%%%%%%%%%%%%%%%

\section{Errors in fitted parameters}
\label{sec:errors-fit}

% Dolphin (2002)
%  merely quoting the ‘best-fitting star formation history’ is useless unless
% one also provides a measurement of the uncertainties.
% \cite{Andrae_2010}
% Any parameter estimate that is given without an error estimate is meaningless
It is well known that a parameter given with no error estimation is
meaningless from a physical standpoint~\citep{Dolphin_2002,Andrae_2010}.
%
% \cite{Paunzen_2006}
% If we look at recent statistical papers dealing with open-cluster
% parameters (e.g. Chen, Hou & Wang 2003; Dias & Lepine 2005),
% none of them have any detailed error treatment included. Although
% apparent ‘mean values’ for the age, distance and metallicity are used,
% the level of accuracy is completely ignored in the final conclusions.
This fact notwithstanding, a detailed error treatment is usually ignored in
articles that deal with star clusters analysis~\citep{Paunzen_2006}.
As explained in Sect.~\ref{ssec:isoch-fit}, \texttt{ASteCA} employs a bootstrap
method to assign standard deviations for each fitted parameter in our OCs
sample.

Since the code must simultaneously fit a large number of free parameters --
five in this case -- within a wide range of allowed values, and using only a
2-dimensional space of observed data (i.e.:\ the $CT_1$
magnitudes)\footnote{We plan to upgrade the code to eventually allow
more than just two observed magnitudes, therefore extending the 2-dimensional
CMD analysis to an N-dimensional one.}, the uncertainties involved will
expectedly be somewhat large.
%
It is worth noting that unlike manually set errors, these are
statistically valid uncertainty estimates obtained via a bootstrap process.
This is an important point to make given that the usual by-eye isochrone fitting
method not only disregards known correlations among all OCs parameters,
it is also fundamentally incapable of producing a valid error
analysis~\citep{Naylor_2006}. Any uncertainty estimate produced by-eye serves
only as a mere approximation, which will often be biased towards smaller
figures. The average logarithmic age error for the OCs in the literature is 0.16
dex, in contrast with the almost twice as large average value estimated by
\texttt{ASteCA} (see below).

\begin{figure}
\includegraphics[width=\columnwidth]{figures/errors_asteca.png}
\caption{\emph{Left}: Distribution of errors versus the five parameters fitted
by \texttt{ASteCA}.\@ Colours are associated to the $CI$ (see bar in top plot),
sizes are proportional to the actual cluster sizes. A small random scatter in
the x axis is added for clarity.
\emph{Right}: Error histogram. The mean error value for each parameter is shown
in the top right corner, and drawn in the plot with a dashed red line.}
\label{fig:errors}
\end{figure}

In Fig.~\ref{fig:errors} we show the distribution of the standard deviations
versus the five fundamental parameters fitted by the code, for the entire sample
(SMC and LMC OCs). Colours follow the $CI$ obtained for each cluster.

The apparent dependence of the metallicity error with decreasing [Fe/H] values
arises from the fact that \texttt{ASteCA} uses $z$ values -- the linear metallicity
measure -- to find the best OC-SC match. The error in $z$ is found to be on
average $e_z{\approx}0.003$ for all analysed clusters\footnote{Approximately
70\% and over 80\% of the OCs in the S/LMC have assigned $e_z{=}0.003$ errors}.
This means that, when converting to $e_{[Fe/H]}$ using the relation

\begin{equation}
e_{[Fe/H]} = e_z/[z*\ln(10)],
\end{equation}

\noindent the $z$ in the denominator makes $e_{[Fe/H]}$ grow as it decreases,
while $e_z$ remains more or less constant.
For very small $z$ values (e.g.: $z=0.0001$), the
logarithmic errors can easily surpass $e_{[Fe/H]}=2\,dex$. In those cases, as
seen in Fig.~\ref{fig:errors}, the error is trimmed to $2\,dex$ which is enough
to cover the entire metallicity range.

Other than the mathematical dependence of the logarithmic metallicity explained
above, there are no visible trends in the arrangement of errors for any of the
fitted parameters. This is a desirable feature for any statistical method.
If the uncertainties of a parameter varied (increase/decrease) with it, it would
indicate that \texttt{ASteCA} was introducing biases in the solutions.

Histograms plotted to the right in Fig. \ref{fig:errors} show the distribution
of errors and their arithmetic means as a dashed red line.
The average metallicity and age errors for the full sample are 0.3 dex and 0.27
dex respectively. For the SMC $e_{[Fe/H]}<0.2$ for $\sim$34\% of the calculated
uncertainties. The same relation is true for $\sim$69\% of the LMC OCs, which
means that OCs in the SMC have a larger dispersion in their assigned
metallicity errors. This is due to the lower overall metal content of the SMC
OCs, combined with the effect described above by which the logarithmic
metallicity error increases for smaller [Fe/H] values.
Approximately 53\% of the combined S/LMC sample show $e_{\log(age/yr)}{<}0.1$, a
very reasonable uncertainty estimate.
Error estimates for the remaining parameters are all within acceptable ranges.\\

Numerically, errors could be lowered applying several different techniques:
increase the number of bootstrap runs, increase the number of models evaluated
in the GA (``generations''), or reduce the value of the steps in the parameters
grid.
%
All of these methods will necessarily extend the time needed to process each
cluster, in particular increasing the number of bootstrap runs. Limited
computational time available requires a balance between the maximum processing
power allocated to the calculations, and the precision one is aiming at. The
error values presented here should then be considered a conservative upper limit
of the accuracy with which the fundamental parameters of our sample can be
obtained by \texttt{ASteCA}.\@




%%%%%%%%%%%%%%%%%%%%%%%%%%%%%%%%%%%%%%%%%%%%%%%%%%%%%%%%%%%%%%%%%%%%%%%%%%%%%%%%
%%%%%%%%%%%%%%%%%%%%%%%%%%%%%%%%%%%%%%%%%%%%%%%%%%%%%%%%%%%%%%%%%%%%%%%%%%%%%%%%

\section{Comparision with published data}
\label{sec:comp-pub-data}

We compare in Sect.~\ref{ssec:lit-values} the resulting parameter values
obtained after running \texttt{ASteCA} on our set of Magellanic OCs, with those taken
from the original reference articles (those that used the same Washington
photometry used in this work, see Table~\ref{tab:literature}) referred as the
``literature''.
%
The parameters age, extinction, and mass are also analysed in
Sect~\ref{ssec:db-values} for a subset of 142 OCs that could be cross-matched
with seven articles (where $UBVRI$ photometry was used, see
Table~\ref{tab:databases}), referred as the ``databases''.

  
%%%%%%%%%%%%%%%%%%%%%%%%%%%%%%%%%%%%%%%%%%%%%%%%%%%%%%%%%%%%%%%%%%%%%%%%%%%%%%%%

\subsection{Literature values}
\label{ssec:lit-values}

\begin{figure*}
\includegraphics[width=2.\columnwidth]{figures/as_vs_lit_S-LMC.png}
\caption[Pass this otherwise it chokes with the cite]{\emph{Left column}:
parameters comparison for the LMC.\@
\emph{Centre column}: idem for the SMC.\@
\emph{Right column}: BA plot with differences in the sense \texttt{ASteCA} minus
literature, for the combined S/LMC sample. For clarity, a small random scatter
is added to both axes for the metallicity and distance modulus plots. Mean and
standard deviation are shown as a dashed line and a grey band, respectively; its
values are displayed in the top left of the plot.
Colours following the coding shown in the bar at the right of the figure, for
each row.~\cite{Piatti_2011b} OCs which contain only age information are plotted
with $E_{B-V}{=}0$ colour coding.}
\label{fig:as_vs_lit}
\end{figure*}

The comparison of \texttt{ASteCA} versus literature values for the parameters
metallicity, age, extinction and true distance modulus, is presented in
Fig.~\ref{fig:as_vs_lit}, one parameter per row. The first two plots show the
1:1 identity line for the LMC and the SMC respectively.
%
The rightmost diagram shows a Bland-Altman (BA) plot for our combined set of
OCs in the MCs, with the variation in the
x axis proposed by Krouwer~\citep{Bland_1986,Krouwer_2008}.\footnote{The BA is
also called a ``difference'' plot or more commonly a ``Tukey Mean-Difference''
plot. In the original BA plot the default x axis displays the mean values
between the two methods being compared. The Krouwer variation changes the means
for the values of one of those methods, called the ``reference''. In our case,
the reference method is \texttt{ASteCA} so we use its reported values in the x
axis.}
%
The BA plot shows differences in the y axis in the sense $\Delta$=(\texttt
{ASteCA} minus literature), versus values found by the code in the x axis. Errors
in $\Delta$ values are calculated combining the errors for both estimates. The
mean of the differences, $\overline{\Delta}$, is shown as a dashed line and its
standard deviation as a grey band.\\

An offset is noticeable for the abundance estimates (first row of
Fig.~\ref{fig:as_vs_lit}), where \texttt{ASteCA} tends to assign values
$\sim$0.22 dex larger than those found in the literature.
On average, this offset is of $\sim$0.27 dex for the SMC and $\sim$0.18 dex for
the LMC.\@
%
This could be explained by two different processes, in light of our
knowledge that the \texttt{ASteCA}'s OC-SC best fit matching introduces no
biases into the solutions.
%
First, the MC's star clusters are generally considered to have low
metallicities. For example, the metal content values assumed in the 19
literature articles, are exactly [Fe/H]=-0.7 and -0.4 for $\sim$60\% and
$\sim$75\% of the S/LMC clusters. The by-eye fit is thus very likely biased
towards the assignment of lower abundances. This type of ``confirmation
bias'' in published values has been studied recently by~\cite{de_Grijs_2014}, in
relation to distance measurements reported for the LMC.\@
Researchers also tend to fit isochrones adjusting it to the lower envelope on an
OC's sequence; which also contributes to the selection of isochrones of
smaller metallicity.\footnote{The reason is that increasing an isochrone's metal
content moves it towards redder (greater colour) values in a CMD, see for
example~\cite{Bressan_2012}, Fig 15.}
%
Second, the code writes an obtained parameter value to file, using as many
significant figures as those given by the rounded standard deviation.
The standard deviation in turn, is rounded following the convention of keeping
only its first significant figure. This means for example that a value of
z=0.0005$\pm$0.00312 will be rounded to z=0.001$\pm$0.003. Although this effect
is an issue only for cases where the uncertainty is larger -- or close to -- the
parameter value, it will nonetheless affect smaller metallicities shifting
them towards larger estimates.\footnote{This behaviour will be changed in the
next release of \texttt{ASteCA}. The parameter estimates will be written to
file keeping more significant figures, to avoid this issue.}
%
These two effects can therefore explain the offset found for the abundances
assigned to OCs in the literature versus those estimated by \texttt{ASteCA}. It
is important to notice that neither effect is intrinsic to the best likelihood
matching method used by the code.

%
There are two SMC OCs which present an extremely low metallicity: NGC 294
($\alpha{=}0^h53^m06^s$, $\delta{=}-73^\circ22'49''$, [J2000.0]) and
HW85 ($\alpha{=}1^h42^m28^s$, $\delta{=}-71^\circ16'45''$, [J2000.0]). These
OCs are positioned close to the SMC's centre and towards its periphery,
respectively.
Both are found to have metallicities of [Fe/H]$\simeq$-2.2 dex -- with large
errors given by \texttt{ASteCA} -- versus their smaller literature values of
-0.7 dex.
The code assigns $\log(age/yr)$ values of $8.8\pm0.06$ for NGC294, and
$9.3\pm0.7$ for HW85; while the respective values in the literature are
$8.51\pm0.47$ and $9.26\pm0.28$.
%
NGC 294 has been assigned abundances as low as -1.2 dex~\citep[see the
integrated spectroscopy study by][]{Dias_2010}, although the value obtained by
\texttt{ASteCA} is substantially lower. HW85 is a little studied cluster with
very few members -- approximately 20 in our photometric data set -- and thus
prone to present variations in its estimated parameters. No other
published estimation of this cluster's metallicity could be found.
%
The sequence of an OC is broadened towards redder colours by the
presence of unresolved binaries. We thus investigate the possibility that our
selected fraction of unresolved binaries ($b_f$=0.5, see
Sect.~\ref{ssec:param-ranges}) could be influencing the selection of these low
metallicities. Both OCs are processed five more times setting the $b_f$
factor to [0., 0.25, 0.5, 0.75, 1.], leaving the metallicity to vary as a free
variable, and fixing the remaining parameters to the values found by
\texttt{ASteCA}'s original run.
% NGC294
% bf = [0., 0.25, 0.5, 0.75, 1.] --> z = [5, 8, 6, 8, 8]x10^(-4) -->
% [Fe/H] = [-1.5, -1.3, -1.4, -1.3, -1.3] --> mean = -1.36+-0.08
% HW85
% bf = [0., 0.25, 0.5, 0.75, 1.] --> z = [5, 8, 5, 6, 2]x10^(-4) -->
% [Fe/H] = [-1.5, -1.3, -1.5, -1.4, -1.9] --> mean = -1.5+-0.2
If we average these results we find that, although in both cases the mean [Fe/H]
is larger than the original value -- -1.36$\pm$0.08 dex and -1.5$\pm$0.2 dex for
the NGC 294 and HW85, respectively --, these are still very low metallicity
estimates. We can thus rule out the binary fraction used as the responsible
factor of these low abundances, and conclude that these are indeed OCs with
low metallicities.
% Spectroscopy studies would be needed to 

The general dispersion between literature and \texttt{ASteCA} values is
quantised by the standard deviation of the $\Delta$ differences in the BA plot.
This value is $\sim$0.32 dex (top left of BA plot), in close agreement with the
mean internal uncertainty found in Sect.~\ref{sec:errors-fit} for this
parameter. Mean metallicity estimates for the MCs using \texttt{ASteCA} values
are $\mathrm{[Fe/H]}_{SMC}{\simeq}-0.52\pm$0.44 dex, and
$\mathrm{[Fe/H]}_{LMC}{\simeq}-0.26\pm$0.24 dex. These are considerably more
metal rich and disperse averages than the ones obtained using literature values:
$\mathrm{[Fe/H]}_{SMC}{\simeq}-0.78\pm$0.23 dex, and
$\mathrm{[Fe/H]}_{LMC}{\simeq}-0.42\pm$0.16 dex\\

The second row in Fig.~\ref{fig:as_vs_lit} shows the age distribution for both
Clouds. The overall agreement is very good, with larger deviations from the
identity line appearing for younger ages. There are ten OCs for which
\texttt{ASteCA} assigned $\log(age/yr)$ values that differ more than 0.5 dex
from their literature values. These are referred to as ``outliers'', and are
treated separately in Sect.~\ref{ssec:outliers}.

In~\cite{Palma_2015b} an offset is found for star clusters in the LMC
fitted with the~\citet[][G02]{Girardi_2002} and~\citet[][B12]{Bressan_2012} set
of isochrones, where OCs fitted with the latter presented consistently larger
values.
%
Here we confirm this trend, for ages assigned to our set of OCs for both the
SMC and the LMC.\@ While \texttt{ASteCA} uses the B12 set of theoretical
isochrones, eleven out of the nineteen articles the literature set 
(see Table~\ref{tab:literature}) used G02 isochrones.
The~\citet[][L01]{Lejeune_2001} and~\cite{Marigo_2008} isochrone sets were used
by four and two literature articles, respectively; with the remaining two
articles applying the $\delta T_1$ index to derive ages.
In~\cite{Piatti_2003b,Piatti_2003a,Piatti_2007a} no significant
differences were found between the L01 and the G02 sets, so we assume the same
offset will be present in the former set.
%
Excluding the outlier OCs described above, the $\Delta \log(age/yr)$ offset for
the S/LMC are $\sim$0.02 dex and $\sim$0.05 dex, respectively.

The mean value for the $\Delta \log(age/yr)$ parameter in the BA plot, including
all S/LMC OCs, is $\sim$-0.01 dex. This points to an excellent agreement among
literature and \texttt{ASteCA} values for the $\log(age/yr)$.
If we exclude the outliers, the mean of the differences increases by a small
amount to $\sim$0.04 dex. Similarly to what was found for the metallicity,
the dispersion between literature and \texttt{ASteCA} values is almost exactly
the internal uncertainty found for errors assigned by the code, i.e.: $\sim$0.3
dex.\\

The reddening distribution is relatively small for OCs in both MCs. Maximum
$E_{B-V}$ values are $\sim$0.15 mag and $\sim$0.3 mag for the S/LMC
respectively, as shown in the \texttt{ASteCA} versus literature identity plots.
The $\Delta$ differences are well balanced with a mean of -0.02 and a
standard deviation of 0.05 mag, slightly larger than the 0.02 mag average
uncertainty found for the errors assigned by the code. We obtain average
$E_{B-V}$ extinctions for the S/LMC of 0.03$\pm$0.03 and 0.05$\pm$0.05,
noticeably lower estimates -- approximately a third -- than those used for
example in the~\cite{Hunter_2003} study of the MCs.\\

The true distance moduli ($\mu_{\circ}$) found by \texttt{ASteCA} in both MCs
show a clear displacement from literature values. This is expected, as the
distance to OCs in the latter is always assumed to be a fixed constant
equivalent to the
distance to the centre of the corresponding galaxy.
The distribution of $\mu_{\circ}$ values found by the code covers the entire
range allowed in Sect.~\ref{ssec:param-ranges}. Distances obtained by \texttt
{ASteCA} thus vary up to $\sim\pm$0.1 mag from the default fixed $\mu_{\circ}$
values used in the literature.
%
It is worth noting that this variation appears to have no substantial effect on
any of the remaining parameters, an effect that could in principle be expected
due to the known correlations between all fundamental parameters (see Paper I,
Sect. 3.1.4).
This reinforces the idea that using a fixed value for the distance modulus, as
done in the literature, is a valid way of reducing the number of free variables
at no extra cost.\\

\begin{figure}
\includegraphics[width=\columnwidth]{figures/as_vs_lit_mass.png}
\caption{\emph{Top}: Mass comparison for \texttt{ASteCA} versus literature
values.
\emph{Bottom}: BA plot with mean and standard deviation of the differences
shown as a dashed horizontal line, and a grey band respectively. Values are
displayed in the top left corner.}
\label{fig:as_vs_lit_mass}
\end{figure}

Since masses are only assigned in~\citet[][M13]{Maia_2013} for its sample of 29
SMC OCs, the comparison with \texttt{ASteCA} is presented separately in
Fig.~\ref{fig:as_vs_lit_mass}.
%
The identity plot (top) shows a trend for \texttt{ASteCA} masses to be smaller
than those from M13. This is particularly true for two of the OCs with
the largest mass estimates -- and the largest assigned errors -- in M13:
H86-97 (3300$\pm$1300$M_{\odot}$), and H86-87
(3100$\pm$1700$M_{\odot}$). The BA plot (bottom) shows that, on average, M13
masses are ${\sim}500{\pm}700\,M_{\odot}$ larger.
%
To explain these differences, we need to compare the way total cluster masses
are obtained by M13 and by \texttt{ASteCA} in this work.

In M13 masses were determined using two methods, both based on estimating
a mass function via the $T_1$ luminosity function (LF) of an OC.\@
In both cases a field star cleaning process was applied. The first one employs
a CMD decontamination procedure \citep[described in][]{Maia_2010}, and the
second one cleans the cluster region's LF by subtracting it a field star LF.\@
The results obtained with these two methods are averaged to generate the final
mass values.
%
Although any reasonable field star cleaning algorithm should remove many or most
-- depending on the complexity of the OC's CMD -- of the foreground/background
stars in a cluster region, some field stars are bound to remain.
For heavily contaminated OCs this effect will be determinant in shaping the
``cleaned'' CMD sequence, as true cluster members will be very difficult to
disentangle from contaminating field stars.
%
The set of 29 OCs in the M13 sample are indeed heavily affected by field star
contamination. This can be seen in Fig.~\ref{fig:as_vs_lit_mass}, where the
colour assigned to each OC corresponds to its $CI$. The minimum value is
$CI{\simeq}0.55$, which means all OCs in the set are expected to contain on 
average more field stars within the analysed cluster region, than true members.
%
The presence of a large number of contaminating field stars not only makes the
job much harder for the DA, it also necessarily implies that the overall LF will
be overestimated, therefore leading -- in the case of M13 --  to an
overestimation of the total mass.
In contrast, \texttt{ASteCA} assigns OC masses taking their values directly from
the best match synthetic clusters (SCs). Field star contamination will thus have
a much lower influence on the code's mass estimate, limited just to how
effective the DA is in cleaning the cluster region.

The case of B48 is worth mentioning, as it is the OC with the largest total mass
given in M13 (3400$\pm$1600$M_{\odot}$).
After being cleaned of possible field stars by the DA -- see
Sect.~\ref{ssec:dencontamination} -- low mass stars are entirely removed and B48
is left only with its upper sequence ($T_1<18.4$ mag).
This happens both in the literature and in the analysis done by \texttt{ASteCA},
see left CMD in Fig.~\ref{fig:B48_DA}, and Fig. C8 in M13.
%
The likelihood defined in Eq.\ref{eq:likelihood} sees then no statistical
benefit in matching the OC with a SC of a similar age and mass,
which will contain a large number of low mass stars.
This leads the GA to select as best match SCs of considerably
younger ages ($\log(age/yr){<}7.0$) than that assigned in M13 ($7.9\pm0.05$),
and with much lower mass estimates (see caption of Fig.~\ref{fig:B48_DA}).
%
% Any other configuration of the DA where the density based removal of stars was
% employed, lead to considerably smaller masses paired with younger age estimates 
% ($\log(age/yr)<7.0$) than those assigned either in M13 ($7.9\pm0.05$) or by the
% code ($7.5\pm0.3$).
%
A good match both in age and in mass could be found by the code, only if the DA
was applied with no cell-by-cell removal of low MP stars as shown in the right
CMD of Fig.~\ref{fig:B48_DA}. This means that all stars within the cluster
region -- including field stars -- are used in the SC matching process, which
inevitably questions the reliability of the total mass estimate.
%
Dealing with this statistical effect is not straightforward and will probably
require an extra layer of modelling added to the SC generation algorithm.
As discussed in Sect.~\ref{ssec:outliers} this effect also plays an important
role in the significant age differences found between \texttt{ASteCA} and the
literature, for a handful of OCs.

\begin{figure}
\includegraphics[width=\columnwidth]{figures/B48_DA.png}
\caption{\emph{Left}: Best fit isochrone for B48 found by \texttt{ASteCA} when
a cell-by-cell removal is applied on its sequence, following the Bayesian MP
assignation (removed stars are drawn semi-transparent). The estimated age and
total mass are $\log(age/yr){=}6.2{\pm}0.6$, and $M{=}400{\pm}200\,M_{\odot}$.
\emph{Right}: Best fit isochrone found when no removal of stars is performed,
and the full cluster region is used in the search for the best synthetic
cluster match. The estimated age and total mass are
$\log(age/yr){=}7.5{\pm}0.3$, and $M{=}3000{\pm}900\,M_{\odot}$.}
\label{fig:B48_DA}
\end{figure}


%%%%%%%%%%%%%%%%%%%%%%%%%%%%%%%%%%%%%%%%%%%%%%%%%%%%%%%%%%%%%%%%%%%%%%%%%%%%%%%%

\subsubsection{Outliers}
\label{ssec:outliers}

Ten of the analysed OCs in this work -- approximately $4\%$ of the total
239 clusters present in the set -- show a difference in age with literature
values of $\Delta\log(age/yr){>}0.5$.
Such a large age difference translates into two very dissimilar isochrones
fitted to the same observed coeval star sequence, which makes this sub-sample of
OCs stand out from the rest.
%
For these ``outliers'' no configuration of the DA plus the employed binning
methods could be found, that resulted in SC matches with age values
more similar to those found in the literature.
In Appendix~\ref{apdx:outliers}, literature and \texttt{ASteCA} isochrone
matches are shown for these ten outliers.

Five of these OCs belong to the LMC and the remaining five to the SMC,
as shown in Table~~\ref{tab:outliers}.
All the OCs in the outliers sample had smaller ages assigned by the code,
compared to their literature values. These differences go from 0.55
dex up to 1.6 dex in the most extreme case of the LMC cluster KMHK975.
%
Fig.~\ref{fig:outliers0} (see CMD diagrams \emph{a}) shows the best matches for
this OC, where the reason for the different isochrones selected can be clearly
seen. While the by-eye isochrone fit done in the literature aligned the brighter
part of the OC's sequence with the turn off point of a $\sim$200 Myr
isochrone, \texttt{ASteCA} decided instead that this was the top portion of a
much younger cluster ($\sim$5 Myr) with no discernible turn off.
%
For almost all of the outliers, the same process can be identified as the main
cause responsible for the observed age differences.
The statistical mismatch due to the removal of low mass stars by the DA
-- discussed in Sect.~\ref{ssec:lit-values} -- can also be seen to affect some
of the fits here. In particular the SMC OCs SL579 and H86-188 show signs of this
effect in the best match SCs selected by \texttt{ASteCA}
(see isochrone fits in Fig.~\ref{fig:outliers0}, CMD diagrams \emph{b} and
\emph{h}).

\begin{table}
\centering
\caption{OCs with large differences in their assigned literature ages versus
the values found by the code (``outliers'').
Equatorial coordinates are expressed in degrees for the $J2000.0$ epoch.
Ages are given as $\log(age/yr)$ for literature (L) and \texttt{ASteCA} (A).
The difference between both estimates (L-A) is given in the last column as
$\Delta$.}
\label{tab:outliers}
\begin{tabular}{lccccc}
\hline
\hline\\[-1.85ex]
Cluster & $\alpha(^\circ)$ & $\delta(^\circ)$ & L & A & $\Delta$\\
\hline\\[-1.85ex]
L-KMHK975 & 82.49583 & -67.87889 & 8.30 & 6.70 & 1.60\\
L-SL579 & 83.55417 & -67.85639 & 8.15 & 7.00 & 1.15\\
L-BSDL631 & 76.64167 & -68.42722 & 8.35 & 7.50 & 0.85\\
L-KMHK979 & 82.41250 & -70.98389 & 7.90 & 7.30 & 0.60\\
L-H88-316 & 85.41250 & -69.22944 & 8.25 & 7.70 & 0.55\\
\\[-1.85ex]
S-L35 & 12.00417 & -73.48611 & 8.34 & 6.90 & 1.44\\
S-H86-188 & 15.05833 & -72.45833 & 8.10 & 6.70 & 1.40\\
S-L39 & 12.32500 & -73.37167 & 8.05 & 7.00 & 1.05\\
S-B134 & 17.25417 & -73.20667 & 8.15 & 7.20 & 0.95\\
S-K47 & 15.79583 & -72.27361 & 7.90 & 7.00 & 0.90\\
\end{tabular}
\end{table}


These age estimates could be improved -- in the sense that they could
be brought closer to literature values -- if a more restrictive age range was
used (e.g.: a minimum value of $\log(age/yr){=}7.5$ instead of 6 dex as used in
this work, see Table~\ref{tab:ga-range}).
Lacking external evidence to substantiate this a priori restriction, we choose
to keep the values obtained by \texttt{ASteCA}, with this section acting as a
cautionary note.


%%%%%%%%%%%%%%%%%%%%%%%%%%%%%%%%%%%%%%%%%%%%%%%%%%%%%%%%%%%%%%%%%%%%%%%%%%%%%%%%  

\subsection{Databases values}
\label{ssec:db-values}

\begin{figure*}
\includegraphics[width=2.\columnwidth]{figures/cross_match_if.png}
\caption{\emph{Left}: age comparison for DBs that used the isochrone fit
method, versus \texttt{ASteCA}, where $N$ is the number of OCs cross-matched in
each DB.\@
\emph{Centre}: ``delta'' plot, showing the differences between extinction
and age, in the sense \texttt{ASteCA} minus DB.\@ Curves represent regions of
iso-densities after fitting a 2-dimensional Gaussian Kernel.
\emph{Right}: same as previous plot, now showing \texttt{ASteCA} minus
literature values for both Clouds.}
\label{fig:cross_match_if}
\end{figure*}

In addition to the analysis performed in Sect.~\ref{ssec:lit-values}, we compare
our results with those taken from seven articles -- the ``databases'' or DBs --
where a different photometric system was used; see Table~\ref{tab:databases}.
%
We can further separate these seven DBs into two groups: those where the
standard by-eye isochrone fitting method was applied -- P99, P00, C06, and G10
-- and those where integrated photometry was employed to derive the OCs
fundamental parameters -- H03, R05, and P12.
%
A total of 142 individual OCs from our sample could be cross-matched. Where
names where not available -- P00, P99, and R05 -- we employed a 20 arcsec radius
to find matches, based on the equatorial coordinates of the OCs in each DB.

%
% Delta (ASteCA - DB) for ages: mean +- std
% P99 Delta diffs ages: -0.13 +- 0.63
% P00 Delta diffs ages: 0.37 +- 0.45
% C06 Delta diffs ages: 0.02 +- 0.58
% G10 Delta diffs ages: 0.23 +- 0.46
%
The comparison of \texttt{ASteCA} ages with those from the four isochrone
fit DBs is shown in Fig.~\ref{fig:cross_match_if}, left and centre plot.
%
P99 and P00 analyse SMC and LMC clusters respectively,
using~\cite{Bertelli_1994} isochrones and fixed metallicities for the S/LMC of
$z{=}0.004, 0.008$. While P99 derives individual reddening estimates based on
red clump stars, P00 use extinction values determined for 84 lines-of-sight
in the~\cite{Udalski_1999} LMC Cepheids study. The mean S/LMC extinction in P99
and P00 are $E_{B-V}{\simeq}0.08{\pm}0.02$ and $E_{B-V}{\simeq}0.14{\pm}0.02$, 
both larger estimates than those found in this work
($E_{B-V}{\simeq}0.03{\pm}0.03$ and $E_{B-V}{\simeq}0.05{\pm}0.05$ for the S/LMC,
see~Sect.\ref{ssec:lit-values}).
%
Both studies attempt to eliminate field star contamination following the
statistical procedure presented in~\cite{Mateo_1986}.
The distance moduli for the S/LMC of $\mu_0{=}18.65, 18.24$ mag employed in
these DBs, are approximately ${\sim}0.25$ mag smaller than the canonical
distances assumed for each Cloud, which has a direct impact on their obtained
ages.
% de Grijs & Anders (2006):
% If they (P00) had assumed a nominal (m-M)=18.50 mag, their resulting age
% estimates would have been younger by log(Age/yr) ~0.2–0.4
% Baumgardt et al. (2012):
% As noted by de Grijs & Anders, this can be explained by the smaller LMC distance
% modulus of (m-M)=18.23 adopted by Pietrzynski & Udalski, compared to the
% value of (m-M)=18.50 used by most other authors. Placing the LMC at a smaller
% distance decreases the absolute luminosities of the stars and the turn-over
% region in the CMD, and therefore makes the clusters appear older.
%
In~\cite{de_Grijs_2006} the authors estimate that had P00 used a value
of $\mu_0{=}18.5$ mag instead for the LMC, their ages would be ${\sim}0.2{-}0.4$
dex younger; a similar conclusion is reached
by~\cite{Baumgardt_2013}.\footnote{Notice that in~\cite{Baumgardt_2013} the
authors correct the age bias that arises in P00 due to the small distance
modulus used, increasing P00 age estimates by 0.2 dex. This is incorrect, ages
should have been decreased by that amount.} The same reasoning can be applied to
the P99 age estimates.\@
%
P99 and P00 ages are displaced on average from \texttt{ASteCA} values (in the
sense \texttt{ASteCA} minus DB) by $-0.13\pm0.6$ dex and $0.37\pm0.5$ dex
respectively, as seen in Fig.~\ref{fig:cross_match_if} left plot.
In the case of P99, the distance modulus correction mentioned above would
bring the age values to an overall agreement with those obtained by
\texttt{ASteCA}, although with a large scatter around the identity line.
%
P00 age values on the other hand, would end up ${\sim}0.7$ dex below the code's
age estimates after such a correction. Such a large deviation is most likely due
to the overestimated extinction values used by P00, as will be shown below.

%
C06 studied 311 SMC clusters via isochrone fitting applying two methods: visual
inspection and a Monte Carlo based $\chi^2$ minimization. The authors also
employ a decontamination algorithm to remove contaminating field stars, making
this the article that more closely resembles this present work.
Distance modulus is assumed to be $\mu_0{=}18.9$ mag. Reddening and
metallicity values of $E_{B-V}{=}0.08$ mag and $z=0.008$ dex are used, adjusted
when necessary to improve the fit.
It is worth noting that the [Fe/H]${=}-0.4$ dex abundance employed in
C06 is closer to the average [Fe/H]${=}-0.52{\pm}0.44$ dex value found for the
SMC by \texttt{ASteCA}, than the canonical value of [Fe/H]${=}-0.7$ used by
default in most works.
This is the DB -- out of the four isochrone-fit and the three integrated
photometry DBs -- that best matches \texttt{ASteCA} age values, with a mean
deviation from the identity line of $0.02\pm0.58$ dex.

%
% \cite{Piatti_2012b}: "The cluster ages derived from isochrone fittings on to
% the cleaned CMDs for 136 clusters, also included in the study of G10, were
% found to be ∼0.2–0.3 in log(age) younger than those derived by G10 [σ ( log
% (age)) = 0.13, age 9.0]. The main reason for this systematic shift is probably
% the different metallicities of the isochrones involved. While C06
% used the isochrone set of Girardi et al. (2002, Z = 0.008), G10 fitted
% the cluster CMDs with isochrones computed by Girardi et al. (1995,
% Z = 0.004).""

% Choudhury et al. (2015); see Fig 5
% Piatti et al. (2015a) - The VMC survey - XV. The Small Magellanic Cloud-Bridge
% connection history as traced by their star cluster populations:
% "a direct comparison of the age estimates derived by Glatt, Grebel & Koch 
% (2010) and our values shows that our ages are \Delta log(t/yr)=0.5 $\pm$ 0.5
% older" ; "Glatt, Grebel & Koch (2010)’s ages are much younger, possibly
% because of the lack of cleaned CMDs and the shallower photometry used by
% these authors."
% Piatti et al. (2015b) - The VMC survey - XVI. Spatial variation of the
% cluster-formation activity  in the innermost regions of the Large Magellanic
% Cloud:
% "Offset is not mentioned in article but it's visible in Fig. 4 middle panel."
G10 analysed over 1500 OCs with ages < 1Gyr in both Clouds via by-eye isochrone
fitting. They assumed distance moduli of (18.9, 18.5) mag, and metallicities of
(0.004, 0.008), for the S/LMC respectively. Extinction was adjusted also by-eye
on a case-by-case basis.
The G10 database presents a systematic bias where smaller logarithmic ages are
assigned compared to our values, with an approximate deviation of $\Delta
\log(age/yr){\simeq}0.23\pm0.46$. This is consistent with the results found
in~\cite{Choudhury_2015} (see Fig. 5), and later confirmed
in~\cite{Piatti_2015a,Piatti_2015b}.
%
G10 does not apply any decontamination method, by which probable field stars are
removed from the cluster's CMD prior to its analysis. Instead, they plot over
the cluster region a sample of surrounding field stars 0.1 arcmin away from the
cluster's radius, taken from a 0.5 arcmin concentric annulus. The lack of a
proper statistical removal of contaminating foreground/background stars can
cause the isochrone fit to be skewed by their presence.

%
As seen in the centre plot of Fig.~\ref{fig:cross_match_if}, these four DBs
taken as a single group present a clear age-extinction bias, when compared
to \texttt{ASteCA} values. This known degeneracy was found in Paper I to be
the one with the largest correlation value (see Paper I, Table 3), meaning it is
the process most likely to affect isochrone fit studies.
The maximum density in this ``delta'' plot is located around
$\Delta E_{B-V}{\simeq-}0.07$ mag and $\Delta \log(age/yr){\simeq}0.3$ dex.
This trend is most obvious for P00 where a rather large average extinction
value was employed, compared to the mean value found by \texttt{ASteCA} as
stated above.
%
For comparison purposes we show in the right plot of
Fig.~\ref{fig:cross_match_if} the same delta plot, this time generated
subtracting literature values from \texttt{ASteCA} age estimations. It can 
be clearly seen that afore mentioned bias is basically non-existent here,
pointing to a consistent overall assignation of extinction and ages by
the code.

In Appendix~\ref{apdx:databases} we show the CMDs of each cross-matched OC for
this four DBs (153 in total). The isochrone plotted is the best fit proposed by
the corresponding DB, and is compared to the best match found by
\texttt{ASteCA}.
Those OCs that present the largest age discrepancies between \texttt{ASteCA} and
DBs values, are those where the same effect mentioned in
Sect.~\ref{ssec:outliers} takes place. A good example of this is SMC-L39, as
seen in Figs.~\ref{fig:DBs_C06_3} and~\ref{fig:DBs_G10_8} for C06 and G10
respectively.\\


% INTEGRATED PHOTOMETRY
%
% Delta (ASteCA - DB) for ages: mean +- std
% H03 Delta diffs ages: 0.44 +- 0.56
% R05 Delta diffs ages: -0.25 +- 0.63
% P12 Delta diffs ages: 0.35 +- 0.44

\begin{figure}
\includegraphics[width=\columnwidth]{figures/cross_match_ip_ages.png}
\caption{Age comparison for DBs that used the integrated photometry
method, versus \texttt{ASteCA}, where $N$ is the number of OCs cross-matched in
each DB.\@}
\label{fig:cross_match_ip_age}
\end{figure}

Our age and mass estimates are also compared with three DBs -- H03, R05, and P12
-- which used integrated photometry to obtain these parameters (see
Table~\ref{tab:databases}). Only two of these, H03 and P12, obtained total
mass values for the OCs in their sample.

H03 studied approximately 1000 OCs in both MCs -- 748 in the LMC and 196 in the
SMC\footnote{To these numbers, 140 and 76 ``questionable'' (according to
H03) S/LMC clusters respectively, can be added to their sample.} --
via $UBVR$ integrated photometry. Ages were assigned based on the Starburst99
model~\citep{Leitherer_1999}, assuming metallicities, distance moduli, and
average $E_{B-V}$ extinction values of (0.004, 0.008), (18.94, 18.48) mag, and 
(0.09, 0.13) mag, for the S/LMC, respectively.
The masses for each OC were derived through their absolute
magnitudes $M_V$ and the mass-luminosity relation, assuming -14.55 mag to be the
$M_V$ of a 10 Myr old $10^6\,M_{\odot}$ cluster with $z=0.008$.
This article represents, as far as we are aware, the largest
published database of MCs cluster masses to date.

R05 used two models -- GALEV~\citep{Anders_2003} and Starburst99 -- combined
with three metallicities values -- (0.004, 0.008) and (0.001, 0.004, 0.008),
used in each model respectively -- to obtain ages for 195 SMC clusters.
This results in five age estimates for each cluster. Individual reddening
values are obtained in the same manner as done in the~\cite{Harris_2004} study,
assigning extinctions according to fixed age ranges.
%
We averaged all extinction-corrected age values for each matched cluster,
and assigned an error equal to the midpoint between the lowest and
highest error bound among all reported ages in the article.

P12 uses the same dataset from H03 to analyse 920 LMC clusters through their
MASSCLEAN$_{colors}$ and MASSCLEAN$_{age}$
packages~\citep{Popescu_2010a,Popescu_2010b}. Ages and masses from duplicated
entries in the P12 sample are averaged.

% % Baumgardt et al. (2013):
% % We average the ages for clusters which appear multiple times in the
% % Popescu et al. sample and are left with a list of 746 unique clusters
% % which are almost identical to those analysed by de Grijs & Anders.
% dG06 finds a systematic difference between the age determinations of H03 with
% their own ages, as well as the ages presented in P00.
% Following \citet[][B13]{Baumgardt_2013}, the dG06 sample is not used given the
% large deviations it presents compared to the G10 and P12 samples.

As seen in the identity plot in Fig.~\ref{fig:cross_match_ip_age}, H03
visibly underestimates ages for younger clusters.
In~\citet[][see Fig. 1]{de_Grijs_2006} this effect was also registered, which
the authors assigned to the photometry conversion done in H03. The average
dispersion between H03 and \texttt{ASteCA} age values is $0.44\pm0.56$ dex.
%
The same happens for P12 ages, albeit with a smaller mean dispersion of
the logarithmic age of approximately $0.35\pm0.44$ dex.
In P12 their own age estimates are compared with those taken from H03.
They find a clear systematic difference with H03 (see P12, Fig. 8),
where MASSCLEAN ages are larger than H03 estimates, for OCs with
$\log(age/yr)<8.$ In our case, most of the OCs cross-matched with P12 are older
than 8 dex. Nevertheless the same trend is confirmed, with P12 age values
located below the identity line in Fig~\ref{fig:cross_match_ip_age} -- i.e.:
younger ages compared to \texttt{ASteCA} -- but still closer than those from
H03. This bias towards smaller age estimates by P12 is consistent to what was
found in~\cite{Choudhury_2015}.
%
Contrary to the what was found for H03 and P12, the R05 study slightly
underestimates ages compared to our results, with a mean dispersion of
$-0.25\pm0.63$ dex around the identity line. The standard deviation is the
largest for the three integrated photometry DBs. In R05 the authors mention
the lack of precision in their age measurements, due to the use of integrated
colours, and the lack of constrains for the metallicity.

% Baumgardt et al. (2012):
% For clusters of a few thousand stars, a single giant star can have a luminosity
% exceeding that of the rest of the cluster, leading to large fluctuations in the
% clusters' M/L ratio (Piskunov et al. 2011). This makes mass estimates based on
% cluster luminosity highly uncertain. We therefore also restrict our analysis to
% clusters more massive than 5000 Mo.
Expectedly, the four isochrone fit studies analysed previously show a more
balanced distribution of ages around the 1:1 relation, in contrast with the DBs
that employ integrated photometry.
%
Ages taken from integrated photometry studies are known to be less
accurate, and should be taken as a rather coarse approximation to the true
values.
As can be seen in P12, integrated colours present large scatters for
all age values, leading inevitably to degeneracies in the final solutions.
The added noise by contaminating field stars is also a key issue, as it is
very difficult to remove properly from integrated photometry data. A
single overly bright field star can also substantially modify the observed
cluster's luminosity, leading to incorrect estimates of its
parameters~\citep{Baumgardt_2013,Piatti_2014_B88}.
A detailed analysis of some of the issues encountered by integrated photometry
studies, and the accuracy of their results, is presented
in~\cite{Anders_2013}.\\

\begin{figure*}
\includegraphics[width=2.\columnwidth]{figures/cross_match_ip_mass.png}
\caption{\emph{Left}: BA mass plot, showing the differences between estimated
masses in the H03 and P12 DBs and the code, in the sense \texttt{ASteCA} minus
DB;\@ symbols as in Fig.~\ref{fig:cross_match_ip_age}.
Only DB masses $\le5000M_{\odot}$ are shown here.
Colours are assigned according to the contamination index (CI) of each OC 
(colorbar is shown in the right plot), sizes are proportional to the actual
sizes in parsecs. The grey band is the mean $\pm 1\sigma$ for the $\Delta M$
values (notice the axis is scaled by $10^{-4}\,M_{\odot}$).
\emph{Centre}: same as previous plot, now showing DB mass values in the range
$5000-20000M_{\odot}$.
\emph{Right}: same as previous plot, for DB mass values $>20000M_{\odot}$.}
\label{fig:cross_match_ip_mass}
\end{figure*}

%
%
% Delta (ASteCA - DB) for mass_DB<5000: mean +- std
% H03 Delta diffs small mass: 36 +- 1661
% P12 Delta diffs small mass: 59 +- 1877
%
% Delta (ASteCA - DB) for mass_DB>5000: mean +- std
% H03 OCs: 18 P12 OCs: 13
% H03 Delta diffs large mass: -37504 +- 83723
% P12 Delta diffs large mass: -19323 +- 26285
%
Masses are obtained in H03 and P12 via integrated photometry analysis.
\cite{Baumgardt_2013} also derives masses, but as stated in that article, their
results are in good agreement with those from P12, so we do not add this
database to our work.

There are 127 OCs in the combined H03 and P12 cross-matched samples.
In Fig.~\ref{fig:cross_match_ip_mass} we show DBs masses for all cross-matched
OCs, versus their relative differences\footnote{For clarity we employ relative
differences here instead of
differences as in previous BA plots, since the masses span a large range of
values. A 5000 $M_{\odot}$ discrepancy will not carry the same weight if it
happens for an OC with estimated masses below 10000 $M_{\odot}$, than if it
happens for an OC with estimated masses above $1{\times}10^5M_{\odot}$.}
defined as:

\begin{equation}
\begin{split}
\overline{\Delta M_r} & = (M_{\mathtt{ASteCA}}-M_{DB})/(M_{\mathtt{ASteCA}}
+M_{DB}) \\
& = \Delta M/(M_{\mathtt{ASteCA}}+M_{DB}),
\end{split}
\label{eq:rel_diffs}
\end{equation}

\noindent where $DB$ represents either H03 or P12, and crossed-matched OCs are
divided into three regions according to the masses given in either database.
%
Sizes are scaled with each OC's radius in parsecs, and colours follow the
difference in assigned ages $\Delta \log(age/yr)$, in the sense \texttt{ASteCA}
minus DB values, as given in the colorbar in the right plot.
The minimum CI value for OCs in this set is ${\sim}0.6$, meaning that they
comprise a highly contaminated sample of clusters.
The grey band in each plot is the mean of the $\overline{\Delta M_r}$ relative
differences in the assigned masses for these OCs -- combining both DBs --,
extended one standard deviation upwards and downwards.

The mean relative differences in the $M_{DBs}\le5000\,M_{\odot}$ low-mass
region -- left plot in Fig.~\ref{fig:cross_match_ip_mass} -- is very close to
zero, with a standard deviation of almost ${\sim}0.5$, equivalent to a
multiplicative factor of 3 between \texttt{ASteca} and DBs masses.
%
The mean value of $\Delta M$, the standard mass difference, is
$\sim40\,M_{\odot}$ which points to a very reasonable scatter around the
identity line. The standard deviation of $\Delta M$ is ${\sim}1700\,M_{\odot}$,
a somewhat large number considering the maximum $5000\,M_{\odot}$ limit for the
OCs analysed. This is nevertheless expected for a set composed of clusters with
such small masses.
As stated for example in P12 and~\cite{Baumgardt_2013}, low mass OCs -- i.e.,
those with $M{\lesssim}$ 5000 or 10000 $M_{\odot}$ -- will tend to have their
estimated integrated photometry masses largely dominated by stochastic
processes.

A surprising systematic trend arises when we take a closer look at the mass
values assigned by the DBs. It can be evidently noticed that the larger the
mass estimated by either DB, the larger the relative difference with 
\texttt{ASteCA}'s derived value. The left plot in
Fig.~\ref{fig:cross_match_ip_mass} already shows this trend, albeit disguised by
some OCs with positive $\overline{\Delta M_r}$ differences --
i.e., with larger masses given by the code -- beyond 2000 $M_{\odot}$.
%
If we look at the centre and right plots in Fig.~\ref{fig:cross_match_ip_mass},
where DBs mass estimates $>5000\,M_{\odot}$ are shown, the trend becomes
unmistakable.
In these plots we see OCs with DB mass estimates up to $1{\times}10^5\,M_{\odot}$
with $\Delta M$ differences of almost $9{\times}10^4\,M_{\odot}$.
The most discrepant case of SMC cluster NGC419 had to be left out of the right
plot in Fig.~\ref{fig:cross_match_ip_mass} for clarity, as it is given a mass of
$3.9{\times}10^5\,M_{\odot}$ by H03 and only $2.8{\times}10^4\,M_{\odot}$ by the
code. This OC is one of the nine clusters -- five in the LMC and four in the SMC
-- identified by H03 as ``extreme'' clusters due to their low absolute magnitude
values (see end of Sect. 4 in H03).

The mean and standard deviation for $\overline{\Delta M_r}$ in the medium and
large-mass regions shown in the centre and right pots of
Fig.~\ref{fig:cross_match_ip_mass}, is ${\sim-}0.5\pm0.3$,
and ${\sim-}0.8\pm0.1$ respectively.
A mass value estimated by the DBs is thus on average between 4 and 9 times
larger than the one calculated by \texttt{ASteCA}.
The smaller standard deviations for $\overline{\Delta M_r}$ in both these
regions, would seem imply that this is not just a stochastic effect but rather a
systematic one.

%
\begin{table*}
\centering
\caption{OCs with large differences ($\Delta M>20000\,[M_{\odot}]$) in their
assigned \texttt{ASteCA} masses, versus the values found in the DBs.
Equatorial coordinates are expressed in degrees for the $J2000.0$ epoch.
Ages are given as $\log(age/yr)$.}
\label{tab:integ_phot_masses}
\begin{tabular}{lcccccccc}
\hline
\hline\\[-1.85ex]
Cluster & $\alpha(^\circ)$ & $\delta(^\circ)$ & H03$_{age}$ &
P12$_{age}$ & \texttt{ASteCA}$_{age}$ &
H03$_{M}\,[M_{\odot}]$ &
P12$_{M}\,[M_{\odot}]$ & \texttt{ASteCA}$_{M}\,[M_{\odot}]$\\
\hline\\[-1.85ex]
%
S-NGC419 & 17.07917 & -72.88417 & $9.31\pm0.12$ & -- & $8.95\pm0.05$ &
$\sim3.9{\times}10^{5}$ & -- & $2.8\pm0.3{\times}10^{4}$\\
%
L-NGC1917 & 79.7583 & -69.001 & $9.48\pm0.09$ & $9.46\pm0.08$ & $9.15\pm0.08$ &
$\sim5.9{\times}10^{4}$ & $1\pm0.05{\times}10^{5}$ & $4\pm1{\times}10^{3}$\\
%
L-NGC1751 & 73.550 & -69.80694 & $9.48\pm0.09$ & $9.06\pm0.01$ & $9.1\pm0.05$ &
$\sim9.7{\times}10^{4}$ & $6.5\pm1{\times}10^{4}$ & $9\pm1{\times}10^{3}$\\
%
S-L27 & 10.35 & -72.89083 & $9.28\pm0.12$ & -- & $9.45\pm0.06$ &
$\sim5.5{\times}10^{4}$ & -- & $1.3\pm0.4{\times}10^{4}$\\
%
L-SL244 & 76.90417 & -68.54194 & $9.48\pm0.09$ & $9.43\pm0.01$ & $9.15\pm0.09$ &
$\sim2.9{\times}10^{4}$ & $3.5\pm0.4{\times}10^{4}$ & $4\pm1{\times}10^{3}$\\
\end{tabular}
\end{table*}

Considering that integrated photometry studies assign more credibility to higher
mass estimates -- as they will tend to be less influenced by stochastic
fluctuations -- this constitutes indeed an unexpected result.
%
In Table~\ref{tab:integ_phot_masses} we show the five OCs with the largest
\texttt{ASteCA}-DBs mass discrepancies, $\Delta M{>}2{\times}10^4\,M_{\odot}$,
ordered locating the ones with the larger DB masses on top.
CMD plots for each one of these OCs are presented in
Appendix~\ref{apdx:largemass}, along with the best match synthetic clusters
generated by the code.
For the most extreme cases, both DBs assign total masses that are over an order
of magnitude greater than the value found by \texttt{ASteCA}.
Ages are largely in good agreement across the two DBs and this work, for these
five OCs.
As mentioned previously for the SMC cluster NGC419, H03 assigns a mass
$3.6{\times}10^5\,M_{\odot}$ -- equivalently 14 times -- larger than the mass
obtained by the code. Similarly, P12 gives LMC cluster NGC1917 a mass 25
times larger -- $\Delta M{=}9.6{\times}10^4\,M_{\odot}$ -- than the one derived by
\texttt{ASteCA}.

A possible explanation for this large divergence in the calculated masses, is
the presence of contaminating field stars.
To test this hypothesis we re-processed with \texttt{ASteCA} all OCs in
Table~\ref{tab:integ_phot_masses}, this time with no previous decontamination
process applied. This means that all stars within the cluster region, including
field stars in the same line of sight, will be employed in the obtention of 
the best synthetic cluster match. The upper limit for the total mass is set to
$5{\times}10^5\,M_{\odot}$, to avoid biasing the results by setting a low total
mass value.
%
For LMC cluster SL244 the total mass value obtained this way is
$3{\times}10^4\,M_{\odot}$, meaning its average DB mass is recovered when no
field stars cleaning is performed.
The best synthetic cluster match for SMC cluster L27 results in an even higher
total mass of $1{\times}10^5\,M_{\odot}$. These two cases clearly highlight the
importance of a proper field star decontamination, before the method to derive an
OC's mass is applied.

For the remaining three OCs -- S-NGC419, L-NGC1917, and L-NGC1751; the most
massive according to H03 and P12 -- the masses derived by \texttt{ASteCA} using
the contaminated cluster region fall short from the values assigned by the DBs.
The closest match is found for LMC clusters NGC1751 and NGC1917, for which a
mass of $2{\times}10^4\,M_{\odot}$ is estimated by the code while their average DB
mass is ${\sim}8{\times}10^4\,M_{\odot}$.
In the case of MC cluster NGC419, the recovered \texttt{ASteCA} mass using the
entirety of stars in its observed field is $5.5{\times}10^4\,M_{\odot}$, still
seven times smaller then the value given by H03. The radius used for NGC 419 in
our case is larger than the one employed in H03 by more than 20\%
(${\sim}85^{\prime\prime}$ versus ${\sim}70^{\prime\prime}$), so we can be sure
that this effect is not related to a lack of stars included within the cluster
region.

These three OCs share a common extra attribute beyond being the ones with
the largest assigned masses in both DBs, and showing the largest discrepancies
with \texttt{ASteCA} mass values: they are all identified as clusters
presenting the controversial ``dual red clump'' (dRC) structure~\citep
{Girardi_2009}.
This feature was predicted in~\cite{Girardi_1998} as a grouping of stars with
enough mass to ignite helium, while avoiding e$^-$-degeneracy.
%
% * made of stars just massive enough to have ignited helium in non-degenerate
% conditions
% * stars just massive enough to ignite He under non-degenerate conditions should
% define a secondary clumpy feature, located about 0.3 mag below the clump of
% lower mass stars, and at its blue extremity
% * He-ignition in stars just massive enough to avoid e-degeneracy settling in
% their H-exhausted cores. The main red clump instead is made of the slightly less
% massive stars which passed through e-degeneracy and ignited He at the tip of
% the red giant branch.
% * simultaneous presence of stars which passed through electron degeneracy after
% central-hydrogen exhaustion and those that did not.
%
%
The quantitative effect of the dRC structure, on the
integrated magnitude of a cluster, was tested on synthetic MASSCLEAN clusters
of varying masses. We found that adding a secondary RC composed of about
${\sim}15\%$ of the stars present in the main RC~\citep[the approximate value
found for NGC 419 in][]{Girardi_2009}, has a very small effect on the synthetic
cluster's integrated V magnitude, as well as its (B-V) colour.\footnote{Where
both V and B filters correspond to the Johnson photometric system.}
%
For a 1 Gyr synthetic cluster of 10000 $M_{\odot}$, adding stars
to form the dRC -- located ${\sim}0.3$ mag below the RC, and
${\sim}0.04$ mag towards the bluer region of the (B-V) vs. V CMD -- results
in the integrated V band (and the integrated (B-V) colour) increasing only
a few hundredths of a magnitude.
At the same time, just doubling the mass of the synthetic cluster, i.e.
20000 $M_{\odot}$, increases the integrated V band value almost a full
magnitude. This difference is large enough to assume that the H03 and P12
models will not mistakenly assign large masses, based on such a small variance
in integrated photometry as that produced by a dRC.\@
%
The excess brightness generated by stars in the dRC region of these three OCs,
would thus appear to not be enough to explain the overestimated total mass
values given by H03 and P12 (in particular to NGC 419 by H03).

In addition to the presence of a dRC, both NGC 419 and NGC 1751 show
extended or multiple main-sequence turnoffs~\citep[MSTO; see:][]
{Glatt_2008,Milone_2009,Rubele_2010,Rubele_2011,Girardi_2011}
while NGC 1917 is known to posses a broadened MSTO~\citep{Milone_2009}.
The origin of this structure is still under debate, as seen in~\cite
{Piatti_Bastian_2016},~\cite{Milone_2016}, and~\cite{Li_2016}.
Its influence on the derived masses from integrated photometry studies is not
straightforward to assess, nor can its impact be easily discarded.\\

\begin{figure*}
\includegraphics[width=2.\columnwidth]{figures/H03_P12_mass.png}
\caption{\emph{Left}: BA plot for the relative difference between P12 minus H03
masses, for average mass values below 5000 $M_{\odot}$. OCs are coloured
according to the difference in their assigned logarithmic ages by each DB --
i.e.: $\Delta \log(age/yr)$ -- in the sense P12 minus H03; see colorbar in the
rightmost plot. The mean and standard deviation for $\overline{\Delta M}_r$ is
shown as a dashed black line and a grey region, respectively.
\emph{Centre}: idem, for average DB masses
$5000<\overline{M}_{DB}<20000\,(M_{\odot})$.
\emph{Right}: idem, for average DB masses
$\overline{M}_{DB}>20000\,(M_{\odot})$.}
\label{fig:h03_p12_cross}
\end{figure*}

If we examine the H03 and P12 databases, we find that the aforementioned bias
-- by which integrated photometric studies increasingly overestimate masses for
larger mass OCs -- exists even when comparing these studies among themselves.
%
After removing duplicated entries in both DBs, and cross-matching them with
a maximum search radius of 20 arcsec, we are left with 670 unique LMC OCs
across H03 and P12.
Fig.~\ref{fig:h03_p12_cross} shows BA diagrams for these cross-matched OCs.
We plot here the average P12-H03 mass $\overline{M}_{DBs}$
versus their relative difference $\overline{\Delta M}_r$, as defined in
Eq.~\ref{eq:rel_diffs}, in the sense P12 minus H03. As done previously, masses
are separated into three regimes for clarity.
%
There are only five OCs that show average masses larger than 100000
$M_{\odot}$, and they are all massive globular clusters which are incorrectly
assigned a low age and mass by P12.\footnote{These five LMC globular clusters
are: NGC 1916, NGC 1835, NGC 1786, NGC 1754, and NGC 1898. P12 assigns masses
below 2000 $M_{\odot}$ in all cases.}
For example, the LMC globular cluster NGC1835 is correctly identified by H03 as
an old ${\sim}5$ Gyr system, with a total mass of
$\sim1.4{\times}10^6\,M_{\odot}$~\citep[a reasonable value,
although a bit overestimated, according to][]{Dubath_1990}. P12 on the
other hand classifies this as an extremely young
${\sim}6.3$ Myr OC with a very low total mass estimate of $1700\,M_{\odot}$.

In Fig.~\ref{fig:h03_p12_cross} we see that, as the average OC mass given by
these DBs increases, so do their relative differences.
The mean values of $\overline{\Delta M}_r$ decreases from ${\sim}0.5\pm0.4$ in
the low mass region, to ${\sim}0.1\pm0.4$ in the medium OC mass region, to
${\sim-}0.4\pm0.6$ in the large mass region. Masses go from being overestimated
a factor of 3 by P12 -- in relation to H03 -- in the low mass region, to being
underestimated by a factor of more than ${\sim}2$ in the large mass region .
%
These differences in total mass grow with larger average masses, in a way that
is closely related to the difference in the $\log(age/yr)$ values estimated by
each DB.\@ Where P12 estimates larger masses than H03 -- i.e., the low mass
region -- it also assigns larger ages by more than 1.5 dex. Conversely, in the
large average mass region, P12 ages can reach differences of up to 3 dex lower
than H03.
%
This age-mass positive correlation, by which an older large cluster can be
incorrectly identified as a much younger and less massive one or vice-versa, is
also noticeable in Fig.~\ref{fig:cross_match_ip_mass} albeit to a lesser
extent. In Fig.~\ref{fig:age_mass_corr} we show how the age-mass correlation
affects DBs estimates compared to those taken from \texttt{ASteCA} (left plot),
and how this effect is much stronger in P12 and H03 estimates (right plot).

\begin{figure}
\includegraphics[width=\columnwidth]{figures/age_mass_corr.png}
\caption{\emph{Top}: Differences plot for \texttt{ASteCA} minus the combined P12
and H03 DBs. Horizontal and vertical axis show differences in
$\log(age/yr)$, and $\log(M/M_{\odot})$, respectively.
A 2-dimensional Gaussian kernel density estimate is shown as iso-density black
curves.
\emph{Bottom}: idem, for ages and masses of P12 and H03 cross-matched OCs, in
the sense P12 minus H03.}
\label{fig:age_mass_corr}
\end{figure}

% H03: 149 con M>5000 de 939 --> 16% 
% H03: 78 con M>10000 de 939 --> 8% 
% H03: 46 con M>20000 de 939 --> 5%
% P12: 156 con M>5000 de 920 --> 17%
% P12: 61 con M>10000 de 920 --> 7%
% P12: 20 con M>20000 de 920 --> 2%
% ASteCA 41 con M>5000 de 239 --> 17% 
% ASteCA 15 con M>10000 de 239 --> 6% 
% ASteCA 6 con M>20000 de 239 --> 2.5% 
About ${\sim}17\%$ of the OCs in H03 and P12, are assigned masses
above 5000 $M_{\odot}$ in their respective DBs. A smaller percentage,
less than ${\sim}8\%$ and ${\sim}5\%$, are assigned by those works mass values
above 10000 $M_{\odot}$ and 20000 $M_{\odot}$ respectively. Similar proportions
are found when inspecting \texttt{ASteCA}'s obtained masses.
%
Clusters with relatively large masses represent -- as demonstrated by
the aforementioned percentages -- a small portion of OCs in the DBs.
Nonetheless, care should be taken when applying their integrated photometry
estimated masses to the study of properties such as the initial cluster
mass function (ICMF).
Large discrepancies in the mass value assigned for the most massive OCs, could
have a non-negligible impact on the slope of the IMCF.\@
\cmmt{AGREGAR ALGO MAS?}




% %%%%%%%%%%%%%%%%%%%%%%%%%%%%%%%%%%%%%%%%%%%%%%%%%%%%%%%%%%%%%%%%%%%%%%%%%%%%%%%%
% %%%%%%%%%%%%%%%%%%%%%%%%%%%%%%%%%%%%%%%%%%%%%%%%%%%%%%%%%%%%%%%%%%%%%%%%%%%%%%%%

\section{Parameters distribution}
\label{sec:param-dist}

We present here a summary of the distribution of fundamental parameter values
obtained with \texttt{ASteCA}, for our set of 239 Magellanic clouds clusters.
%
% \cite{Piatti_2014_prob} analysed XX small angular sized candidate clusters in
% the LMC and found that $13\pm6 \%$ of these catalogued systems were possible
% non-clusters. \texttt{ASteCA} includes a function which obtains the probability
% for a given star region of having arisen from the same distribution as its
% surrounding field. About $5\%$ (13) of our total sample had probabilities
% assigned in the lower quarter, i.e. below $25\%$. This means that the
% likelihood
% that these are real star clusters is very low according to the code.

% * Duong probability for SMC
% Probs<0.25: 7
% Clust H86-70, prob: 0.14
% Clust BS88, prob: 0.22
% Clust H86-197, prob: 0.20
% Clust B111, prob: 0.20
% Clust H86-85, prob: 0.14
% Clust H86-90, prob: 0.16
% Clust NGC242, prob: 0.23
% * Duong probability for LMC
% Probs<0.25: 6
% Clust HS151, prob: 0.18
% Clust KMHK58, prob: 0.13
% Clust H88-52, prob: 0.16
% Clust OGLE298, prob: 0.11
% Clust H88-245, prob: 0.23
% Clust H88-316, prob: 0.13

In astrophysics analysis, histograms are widely used to derive a large number of
properties when a substantial amount of data is available. A galaxy's star
formation history (SFH) is a good example of such a property, almost always
quantified via a one-dimensional histogram.
%
Their widespread use notwithstanding, the generation of a histogram is affected
by well known statistical issues;
see~\cite{Silverman_1986},~\cite{Simonoff_1997}. Different selected bin widths
and anchor positions (point of origin) can make histograms built
from the exact same data look utterly dissimilar. In the worst cases, completely
spurious sub-structures may appear, leading the analysis towards erroneous
conclusions
%
We bypass these issues by constructing an adaptive (or variable) Gaussian kernel
density estimate (KDE) in one and two dimensions, using the standard deviations
associated to a given parameter as the bandwidth estimates. The formulas for
both KDEs are:

\begin{equation}
KDE_{1D}(x) = \frac{1}{N\sqrt{2\pi}} \sum_{i=1}^N \frac{1}{\sigma_i}
e^{-\frac{(x-x_i)^2}{2\sigma_i^2}},
\label{eq:kde-1d}
\end{equation}

\begin{equation}
KDE_{2D}(x,y) = \frac{1}{2\pi N} \sum_{i=1}^N \frac{1}{\sigma_{xi}\sigma_{yi}}
e^{-\frac{1}{2} \left( \frac{(x-x_i)^2}{\sigma_{xi}^2} + 
\frac{(y-y_i)^2}{\sigma_{yi}^2} \right)},
\label{eq:kde-2d}
\end{equation}

\noindent where $N$ is the number of observed values, $x_i$ is the ith observed
value of parameter $x$, and $\sigma_{xi}$ its assigned standard deviation (same
for $y_i$ and $\sigma_{yi}$). The 1D version of these KDEs is similar to the
``smoothed histogram'' used in the~\cite{Rafelski_2005} study of SMC clusters.
%
The use of standard deviations as bandwidth estimates means that the
contribution to the density map (in 1D or 2D) of parameters derived with large
errors, will be smoothed (or ``spread out'') over a large portion of the
parameter's domain. Precise parameter values on the other hand -- i.e. those
with small assigned errors -- will contribute to a much more narrow region, as
one would expect.

\begin{figure*}
\includegraphics[width=2.\columnwidth]{figures/as_kde_maps0.png}
\caption{One and two-dimensional Gaussian adaptive KDEs for the age, metallicity
and mass parameters. Top and right plots are 1D KDEs while the centre plots are
2D KDEs.
Observed clusters are plotted as red and blue stars for the S/LMC,
respectively in the 2D KDEs. Sizes are scaled according to each OC's radius in
parsecs. A small scatter is introduced for clarity.}
\label{fig:kde_fig0}
\end{figure*}

\begin{figure*}
\includegraphics[width=2.\columnwidth]{figures/as_kde_maps1.png}
\caption{Same as Fig.~\ref{fig:kde_fig0} for the extinction and distance
modulus parameters.}
\label{fig:kde_fig1}
\end{figure*}

Replacing one and two-dimensional histogram analysis with these KDEs has two
immediate benefits: a- it frees us from having to select an
arbitrary value for the bandwidth (the most important component of a KDE,
equivalent to the bin width of a regular histogram), and b- it naturally
incorporates the errors obtained for each studied parameter into its derived
probability density function.
Figs.~\ref{fig:kde_fig0} and~\ref{fig:kde_fig1} show 1D and 2D density maps
constructed via Eqs.~\ref{eq:kde-1d} and~\ref{eq:kde-2d} for two paired
parameters, for each of the MCs.\@
%
Being probability density functions means that the area under the curve
integrates to 1. This makes the distributions for a given parameter comparable
for both galaxies, even if the number of observed points -- clusters in our case
-- is not the same (equivalent to a normalized histogram).
%

Based on the analysis of a database of over 1500 OCs, G10 reports two periods of
enhanced cluster formation in thr MCs: around 160 Myr and 630 Myr for the SMC,
and around 125 Myr and 800 Myr for the LMC.\@ A third period is reported at
approximately 8 Myr for both MCs.
%
To obtain these results, the authors construct histograms of their data,
combined with data from P00 and C06 (see Figs. 5 and 6 in G10). They analyse the
peaks found in both histograms and conclude that the formation episodes are
correlated, as they happened around the same period of time.
%
This is a good example of the issues mentioned at the beginning of this section,
regarding histogram construction. If we look at the 1D $\log(age/yr)$ KDEs in
Fig.~\ref{fig:kde_fig0} (top), we see that most of the enhanced formation
periods reported in G10 are not present.
A distinct period of cluster formation is visible in the LMC starting around the
${\sim}$5 Gyr mark, which lasted up to ${\sim}$1.3 Gyr ago.
A similar, but much less pronounced peak is seen for the SMC, with a clear drop
in cluster formation around ${\sim}$2 Gyr.
The height difference  between the SMC and LMC KDEs is related to the decline in
cluster formation. While the LMC sharply drops to almost zero from ${\sim}$1 Gyr
to present times, the SMC shows a much softer descent with smaller peaks
around ${\sim}$250 Myr and ${\sim}$130 Myr.
The well known ``age gap'' in the LMC between 3--10 Gyrs~\citep{Balbinot_2010}
is present, visible as a marked drop in the $KDE_{\log(age/yr)}$ curve at
approximately $\sim9.5$ dex. No clusters older than this age are found in our
work.

The 2D KDE age-metallicity map shows how spread out these values are for OCs in
the SMC, compared to those in the LMC which are much more heavily
clustered together.
Although in this map the abundance of the SMC can be seen to reach
substantially lower values than the LMC, the 1D KDE to the right reveals that
the [Fe/H] parameter peaks between 0 dex and -0.2 dex, for OCs in both clouds.

The age-mass 2D map shows a clustering around younger ages and smaller masses
for the LMC, relative to the SMC.\@ The OC seen in the bottom right corner of
the age-mass SMC map is HW42 ($\alpha{=}1^h01^m08^s$, $\delta
{=}-74^\circ04'25''$, [J2000.0]), a small cluster ($r_{cl}{<}20$ pc) located
close the the SMC's centre. Though its position in the map is somewhat
anomalous, the 1$\sigma$ error in its age and mass estimates could move it to
$\log(age/yr){\simeq}9.4$ and $\log(M/M_{\odot}){\simeq}2.6$. This OC is
classified as a possible emissionless association by~\cite{Bica_1995}.
%
In both clouds there's a tendency for the mass, and the size of the OC, to grow
with the estimated age, as expected.
In the 1D mass KDE, we see that the LMC accumulates most OCs in the low mass
regime (${\sim}3000\,M_{\odot}$).
This is also true for the SMC which has a larger proportion of large mass OCs,
with a distinctive peak around ${\sim}30000\,M_{\odot}$.


% dm +- std: 18.96 +- 0.08
% dm +- std: 18.49 +- 0.08
As seen in Fig.~\ref{fig:kde_fig1} (top), the 1D KDEs of the true distance
moduli are well behaved and clearly normal in their distribution.
A Gaussian fit to these curves results in best fit values of 18.96$\pm$0.08 mag
for the SMC, and 18.49$\pm$0.08 mag for the LMC.\@ The initial ranges given in
Table~\ref{tab:ga-range}, as well as the literature mean distances, are thus
properly recovered.
%
Extinction values are much more concentrated in the SMC around $E_{B-V}
{\approx}0.015$ mag. The OCs in the LMC on the other hand, show that most
values are dispersed below $E_{B-V}{\approx}0.1$ mag, with a shallower peak
located at ${\sim}0.03$ mag.


% %%%%%%%%%%%%%%%%%%%%%%%%%%%%%%%%%%%%%%%%%%%%%%%%%%%%%%%%%%%%%%%%%%%%%%%%%%%%%%%%

% \subsection{Cluster radius}
% \label{ssec:core-radius}

% In Fig~\ref{fig:rad_pars} the distribution of metallicity and ages for both MCs
% versus the assigned radii, is shown.
% \cite{Wilkinson_2003}.

% \begin{figure}
% \includegraphics[width=\columnwidth]{figures/rad_vs_pars_S-LMC.png}
% \caption{XXXXX}
% \label{fig:rad_pars}
% \end{figure}


% %%%%%%%%%%%%%%%%%%%%%%%%%%%%%%%%%%%%%%%%%%%%%%%%%%%%%%%%%%%%%%%%%%%%%%%%%%%%%%%%

\subsection{Age-metallicity relation}
\label{ssec:amr}

A stellar system's age-metallicity relation (AMR) is an essential tool to learn
about its chemical enrichment evolution.
This relation is usually presented either as scattered single points in the
age-metallicity space, or as a function created by grouping and averaging
metallicity estimates in arbitrary age bins.
%
In \cite{Piatti_2010_AMR} a method was devised to generate an AMR able to take
into account the errors in age values, to produce bins of different sizes. This
method has been applied to the obtention of AMRs in~\cite{Piatti_Geisler_2013},
and also adapted to derive star cluster frequency distributions~\citep[e.g.,][]
{Piatti_2013_CF}.

We can take advantage of the KDE technique described in
Sect.~\ref{sec:param-dist} -- used to produce two-parameters density maps -- to
generate an AMR that is truly representative of the observed data, with some
important improvements over previous methods.
%
First, a Gaussian density map has no dependence on the number, size 
(fixed or variable) or location of bins, as a regular histogram or the
aforementioned method would.
Second, the errors in the two parameters used to obtain the density map (age and
metallicity), are organically included in the function that generates it
(as explained in Sect.~\ref{sec:param-dist}).
This means that no ad-hoc procedure is required to incorporate the important
information carried by these values, into the final AMR.\@

The process of creating an AMR function\footnote{By ``AMR function'' we mean a
curve that spans the entire observed age range, mapping each age value
to a single metallicity value.} requires that we associate a unique [Fe/H] to a
single age value, for the available age range.
%
After generating the age-metallicity 2D density map, a dense 2D grid is created
dividing it into N steps of 0.01 dex width, covering the ranges of both
parameters. Every point in this grid is evaluated in the KDE map and its value 
($w_i$) is stored, along with its age-metallicity coordinates ($age_i,\;
[Fe/H]_i$).
Each of the N ages defined in the grid is then associated to a single
representative [Fe/H] value. This representative metallicity for a given age is
obtained as the mean metallicity value, weighted by the KDE function at that
particular age. The formal equation for obtaining it, can be written as

\begin{equation}
\overline{[Fe/H]}_{age_i}=\frac{\sum w_i {[Fe/H]}_i}{\sum w_i}
\label{eq:w-feh}
\end{equation}

\noindent where the summations are performed over each $i$ step for the N
defined steps in the metallicity range, ${[Fe/H]}_i$ is the metallicity value
at step $i$, and $w_i$ is the value of the 2D KDE map for that fixed age and
metallicity coordinates.
The $age_i$ subindex in Eq.~\ref{eq:w-feh} indicates that this mean
metallicity was calculated for a fixed age value, and thus represents a unique
point in the AMR.\@
A similar version of this weighted metallicity was employed 
in~\citet[][see Eq. 3]{Noel_2009} to derive AMR estimates for three observed
fields.
We apply the above formula to all the N ages in the grid defined at the
beginning of the process. The standard deviation for each $\overline{[Fe/H]}_
{age_i}$ value is calculated trough the equation

\begin{equation}
\sigma_{age_i}^2=\frac{\sum w_i \sum [w_i {({[Fe/H]}_i -
\overline{[Fe/H]}_{age_i})}^2]}{{(\sum w_i)}^2 - \sum w_i^2}
\label{eq:w-std-dev}
\end{equation}

\noindent where again all summations are applied over N, and the descriptions
given for the parameters in Eq.~\ref{eq:w-feh} apply.
%
At this point, this method already gives us an AMR function estimate, since
every age step is mapped to a unique metallicity. The downsides are that
the AMR is noisy due to the very small step of 0.01 dex used, and that the
associated errors are quite large.
This latter effect arises because the weighted standard deviation, Eq.~\ref
{eq:w-std-dev}, will be affected not only by the errors in both measured
parameters -- through the constructed 2D KDE map -- but also by the intrinsic
dispersion in the metallicity values found for any given age.
%
To solve this, we calculate the average [Fe/H] for a given age interval, rather
than assigning a metallicity value to each age step in the grid. Dividing the
age range into intervals requires a decision about the step width, much like
when constructing a histogram, bringing back the issue of binning.
We have two advantages here: a- we use Knuth's algorithm (see
Sect.~\ref{ssec:isoch-fit}) to obtain the optimal binning for our data, and
b- the final AMR function is very robust to changes in the binning method
selected, so even the previous choice is not crucial in determining the
shape of our AMR.\@
%
Finally, the $\overline{[Fe/H]}_{age_i}$ values obtained for every ${age_i}$
within a defined age interval, are averaged. Errors are propagated through the
standard formula, disregarding covariant terms~\citep[][Eq. 3.14]
{Bevington_2003}.

The final \texttt{ASteCA} AMRs for the S/LMC can be seen in Fig.~\ref{fig:amr}
as red and blue continuous lines, respectively. Stars show the position of all
OCs in our sample for each galaxy, with sizes scaled according to their radii.
%
The shaded regions represent the 1$\sigma$ standard deviations of the AMR
functions. These regions span a [Fe/H] width of approximately 0.2 dex for both
Clouds, for the entire age range.
%
The blue (top) and red (bottom) vertical segments in the top plot are the bin
edges determined for each age interval by Knuth's algorithm, for the LMC and the
SMC respectively.
% The last blue segment visible around ${\sim}5.1$ Gyr is added to
% include points in the grid that extend beyond the most extreme OC ages.
% This is due to the age error associated to each OC, which makes the KDE map
% evaluate to non-null values farther away from where OCs are positioned.
% Similarly, an edge is added at the left of the age range but it is not
% shown; the same is true for the SMC.\@
%
As stated previously, the final AMR functions are mostly unaffected by the
chosen binning method. Using Knuth's algorithm results in approximately 10 age
intervals of 1 Gyr width, and the AMRs seen in Fig.~\ref{fig:amr}.
If instead we use 100 intervals of ${\sim}0.1$ dex width, the only substantial
change is that the SMC curve is raised by ${\sim}0.1$ dex, for ages below 500
Myr.
%
The two SMC clusters with extremely low metallicities -- [Fe/H]${<-}2$
dex -- are NGC294 and HW85;\@ their abundances were already analysed in
Sect.~\ref{ssec:lit-values}.
Having such small metallicity values means that their associated uncertainty
will be quite large, as discussed in Sect.~\ref{sec:errors-fit}.
This has the effect of spreading their positions in the density
map, preventing these low values from affecting the AMR substantially.
If these two OCs are excluded from our data, the resulting AMR for the SMC moves
upwards in the [Fe/H] axis by less than 0.02 dex.

Several chemical evolution models and empirically estimated AMRs can be found in
the literature for both Magellanic Clouds. Many of the studies performed on the
age-metallicity relation of the S/LMC present their results as scattered points
in the age vs metallicity plot, rather than fitting a unique AMR function to
describe the distribution of those observed values.
%
To allow a straightforward comparison with our AMR functions, we show in
Fig.~\ref{fig:amr} -- centre and bottom plots -- the
functions presented in twelve other works.
These studies constitute a representative sample of the different methods and
data used in the literature over the past twenty years, for these two
galaxies:
\citet[][PT98; bursting models]{Pagel_1998}, \citet[][G98; closed-box model
with Holtzman SFH]{Geha_1998}, \citet[][HZ04]{Harris_2004}, \citet[][C08a;
average of four disk frames]{Carrera_2008_lmc}, \citet[][C08b; average of
thirteen frames]{Carrera_2008_smc}, \citet[][HZ09]{Harris_2009}, \citet[][N09;
5th degree polynomial fit to the AMRs of their three observed regions]
{Noel_2009}, \citet[][TB09; 1: no merger model, 2: equal mass merger, 3: one
to four merger]{Tsujimoto_2009}, \citet[][R12; four tiles average]{Rubele_2012},
\citet[][C13; B: Bologna, C: Cole]{Cignoni_2013}, \citet[][PG13]
{Piatti_Geisler_2013}, and \citet[][M14; 0: field LMC0, 1: field LMC1 , 2: field
LMC2]{Meschin_2014}.
Details on how these AMRs were constructed will not be given here, as
they can be consulted in each reference.
%
All of the above mentioned articles used field stars for the obtention
of their AMRs. This is, as far as we are aware, the first work were the AMR
function for both galaxies is derived entirely from observed star clusters.

% PT98: bursting models for both galaxies.
% G98: closed-box model, Holtzman
% C08 LMC: global average of 13 frames, .
% C08 LMC: average of 4 frames, for the disk.
% TB09: 3 models, M1 (no merger), M2 (1:1 merger), M3 (1:4 merger)
% C13: B: Bologna code, C: Andre Cole's annealing procedure
% M14: LMC0 (R$_{gc}{=}7.1^{\circ}$), LMC1 (R$_{gc}{=}5.5^{\circ}$),
% LMC2 (R$_{gc}{=}4.0^{\circ}$)

\begin{figure}
\includegraphics[width=\columnwidth]{figures/AMR_asteca.png}
\caption{\texttt{ASteCA}'s age-metallicity relation for the SMC (red solid line)
and the LMC (blue solid line). Shaded areas are the 1$\sigma$ regions for each
AMR.\@ Blue and red stars are the LMC and SMC clusters in our set. See
Sect.~\ref{ssec:amr} for more details.}
\label{fig:amr}
\end{figure}

The overall trend of both AMRs coincides with what has already been found in the
literature, namely that the metallicity increases rather steadily with younger 
ages. On average, our AMR estimates are displaced slightly towards more metal
rich values, particularly in the case of the SMC.\@ This tendency was already
mentioned in Sect.~\ref{ssec:lit-values}, where possible causes for at least
some part of the effect were also given.

For the LMC galaxy, Fig.~\ref{fig:amr} centre plot, we see a marked
drop in metallicity from ${\sim-}0.3$ dex beginning around 3.8 Gyr, and
ending 3 Gyrs ago at ${\sim-}0.55$ dex. The former high [Fe/H] value is not
found in any of the remaining functions for the LMC.\@ The M14-0 curve seems to
reproduce this behaviour, but shifted ${\sim}0.8$ Gyr towards younger ages, and
with a much less pronounced peak that reaches its maximum of ${\sim-}0.5$ dex at
${\sim}3.2$ Gyr.
%
The high metallicity value in the LMC AMR at 3.8 Gyr, is caused by the four OCs
located around ${\sim}3.3$ Gyr with [Fe/H]${\approx-0.25}$ dex; when they are
excluded, the peak disappears.
These four OCs are: SL33, H3, SL5, and KMHK586, their CMDs are shown in
Appendix~\ref{apdx:amr_lit}, Fig~\ref{fig:largemet}.
The last three of these OCs are analysed in~\citet{Piatti_2011b} and have no
metallicities or extinction values assigned,
as only ages were estimated in this article via the $\delta T_1$ index.
Ages obtained by \texttt{ASteCA} for this group of OCs are between 0.1-0.3 dex
larger than the ones given in~\citet{Piatti_2011b}.
According to this article, the oldest OCs in the group have ages of ${\sim}2.5$
Gyr. This means that if literature's ages were used instead of the ones produced
by the code, the peak would not show.
In fact, the entire AMRs functions derived using literature values -- i.e.,
those taken from the articles in Table~\ref{tab:literature} -- are markedly
different from the ones shown in Fig.~\ref{fig:amr}.
This is expected as most OCs in these works are assigned fixed metallicities of
-0.7 dex and -0.4 dex for the S/LMC, respectively; particularly for estimated
ages below 1 Gyr. Literature AMRs for both MCs can be seen in
Appendix~\ref{apdx:amr_lit}, Fig~\ref{fig:amr_lit}.

% LMC clusters with large ASteCA ages and met values:
% Name (RA, DEC): (ASteCA, Literature)
% H3 (05:33:20, -68:9:8): (9.5, 9.4); (-0.18, --)
% SL5 (04:35:38, -73:43:54): (9.55, 9.4); (-0.18, --)
% SL33 (04:46:25, -72:34:06): (9.55, 9.3); (-0.28, -0.4)
% KMHK586 (05:08:51, -67:58:49): (9.55, 9.26); (-0.18, --)

%
After the aforementioned drop in the LMC's AMR, there is a steep climb from 3
Gyr to 2 Gyr reaching almost [Fe/H]${\sim-}0.25$ dex, and then a sustained but
much more shallow increase up to the present day's metal content of ${\sim-}0.2$
dex. The AMR functions for the LMC that differ the most from the one obtained
with \texttt{ASteCA} values, are those taken from HZ09 and G98. These two curves
are visibly separated, not only from our AMR, but also from the rest of the
group.
Our average metallicity value for present day OCs, coincides reasonably well
with those from PT98, C08a, and M14-2. The PT98 bursting model and field 2
from M14, show nonetheless a very different rate of increase from 2 Gyr to
present times, compared to \texttt{ASteCA}'s AMR. The average AMR from C08a,
while lacking finer details, provides a better match for this age range.

%
The AMR function for the SMC obtained using \texttt{ASteCA}'s age and
metallicity values, is shown along ten AMRs taken from the published literature
in Fig.~\ref{fig:amr}, bottom.
%
Our AMR shows an increased rate of [Fe/H] that reaches a peak around
${\sim}7.5$ Gyrs ago, followed by a marked dip between $6{-}7$ Gyr.
This feature was reported by TB09 in its two merger models.
The TB09-2 model (equal mass merger) appears to best mimic our AMR for the SMC,
although ours is shifted upwards towards higher abundances. The TB09-3 model 
(1:4 mass merger) shows the same peak, but followed by a much more pronounced
dip that goes well below most AMRs, including our own.
%
The TB09-2 model and our AMR follow a very similar path up to 3 Gyr,
where they start to deviate. For present day ages, where the AMRs differ the
most, our curve is more metal rich by about 0.4 dex compared to the TB09-2
model.

Following the dip beyond ${\sim}6$ Gyr, abundances in our AMR plateau around a
value of [Fe/H]${\simeq-}0.8$ dex until approximately 3 Gyrs ago, where the
rate of growth for the metallicity increases considerably. From that point up to
the present day, the average metallicity for OCs in the SMC grows by about 0.4
dex, according to the \texttt{ASteCA} AMR.
%
The increased rate of growth behaviour for ages younger than 3 Gyrs, is only
reproduced by the PT8 model, and the HZ04 function. In both cases, the maximum 
[Fe/H] value attained by these AMRs for the youngest ages is lower by
${\sim}0.12$ dex, compared to our own estimate for this galaxy.\\

%
The canonical metallicity values usually accepted -- [Fe/H]${=-0.7}$ dex,
[Fe/H]${=-}0.4$ dex, for the S/LMC -- are shown to be off by ${\sim0.3}$ dex on
average, for clusters younger than 2.5 Gyr in both galaxies.
%
Overall our AMRs can not be explained by any single model or empirical AMR
function, and are best reproduced by a combination of several of these. A
similar result was found in~\cite{Piatti_Geisler_2013}, although their AMRs --
derived from field star population -- are significantly different from ours,
mainly for the SMC case.

It is important to remember that the AMRs estimated using the ages and
metallicities derived via \texttt{ASteCA}, are averaged over the structure of
both galaxies. In Fig.~\ref{fig:ra-dec} we showed that our set of OCs covers a
large portion of the surface of these galaxies.
%
If more OCs where available so that the AMRs could be instead
estimated by sectors in the S/LMC, it is entirely possible that different
results would arise. When clusters in our set are divided by sectors,
statistically low numbers are assigned to each -- particularly for the SMC --
which makes this more detailed study not feasible.\\

\cmmt{AGREGAR/MODIFICAR ALGO?}




%%%%%%%%%%%%%%%%%%%%%%%%%%%%%%%%%%%%%%%%%%%%%%%%%%%%%%%%%%%%%%%%%%%%%%%%%%%%%%%%
%%%%%%%%%%%%%%%%%%%%%%%%%%%%%%%%%%%%%%%%%%%%%%%%%%%%%%%%%%%%%%%%%%%%%%%%%%%%%%%%

\section{Summary and conclusions}
\label{sec:summ-concl}

We presented an homogeneous catalogue for a set of 239 star clusters in the
Large and Small Magellanic Clouds, observed with the Washington photometric
system. The OCs span a wide range in metallicity, age, and mass, and are
spatially distributed throughout both galaxies.
%
The fundamental parameters metallicity, age, reddening, distance modulus, and
total mass were determined using the \texttt{ASteCA} package.
%
This tool allows the automated processing of an OC's positional and photometric
data, resulting in estimates of both its structural and intrinsic/extrinsic
properties.
%
As already shown in Paper I, the advantages of using this package include
reproducible and objective results, along with a proper handling of the
uncertainties involved in the synthetic cluster matching process.
%
This permits the generation of a truly homogeneous catalogue of observed
clusters, with their most important parameters fully recovered.
%
Our resulting catalogue is complete for all the analysed parameters, including
metallicity an mass, two properties often assumed or not obtained at all.

% TOMADO DE BAAA58, MODIFICAR v
The analysis of our results, demonstrate that the assigned values for the OCs
are in good agreement with published literature which used the same
Washington photometry.
%
The metallicity parameter showed to be the most discrepant one, with \texttt
{ASteCA} values being on average ${\sim}$0.22 dex larger then those present in
the same photometry literature. The most likely explanation for this is the
confirmation bias effect, by which most OC studies will assume the canonical 
[Fe/H] values rather than derive them through statistically valid means.
%
We also compared our results with those taken from articles that used different
photometric systems. While the differences in age in this case are somewhat
larger, they can be mostly explained by effects outside the code.
% TOMADO DE BAAA58, MODIFICAR ^
Internal errors show no biases present in our determination of fundamental
parameters.

We performed a detailed comparative study of masses obtained through integrated
photometry studies, with our own estimates from CMD analysis. The total mass
values from the latter studies are shown to be systematically overestimated.
This result is opposite to the expectation that larger clusters can have their
masses recovered with moderate accuracy, via integrated photometry.

A method for deriving the distribution of any fundamental parameter -- or a
combination of two of them -- is presented. This method takes into account the
information contained by the uncertainties, often excluded from the analysis.
By relying on Gaussian kernels, it is robust and independent of ad-hoc binning
choices.
%
An age-metallicity relation is derived using the above mentioned method and the
parameter values obtained by \texttt{ASteCA}, for both galaxies. The AMRs
generated can not be fully matched by any model or empirical determination found
in the recent literature.

We demonstrated that the \texttt{ASteCA} package is able to produce proper
estimations of observed star clusters, with their fundamental parameters
covering a wide range of values. A necessary statistically valid error analysis
can be performed, thanks to its built-in bootstrap error assignment method.
%
The tool is also proven to be capable of operating almost entirely unassisted,
on large databases of clusters. This is an increasingly essential feature
of any astrophysical analysis tool, given the growing importance of big data and
the necessity to conduct research on large astronomical data sets.




%%%%%%%%%%%%%%%%%%%%%%%%%%%%%%%%%%%%%%%%%%%%%%%%%%%%%%%%%%%%%%%%%%%%%%%%%%%%%%%%
%%%%%%%%%%%%%%%%%%%%%%%%%%%%%%%%%%%%%%%%%%%%%%%%%%%%%%%%%%%%%%%%%%%%%%%%%%%%%%%%

\section*{Acknowledgments}

GIP would like to thank the help and assistance provided throughout the
redaction of several portions of this work by: D. Hunter, A. E. Dolphin,
M. Rafeslski, D. Zaritsky, T. Palma, F. F. S. Maia, B. Popescu,
and H. Baumgardt.
%
This research has made use of the
VizieR\footnote{\url{http://vizier.u-strasbg.fr/viz-bin/VizieR}} catalogue
access tool, operated at CDS, Strasbourg, France~\citep{Ochsenbein_2000}.
%
This research has made use of
``Aladin sky atlas''\footnote{\url{http://aladin.u-strasbg.fr/}} developed at
CDS, Strasbourg Observatory, France~\citep{Bonnarel2000,Boch2014}.
%
This research has made use of NASA's Astrophysics Data
System\footnote{\url{http://www.adsabs.harvard.edu/}}.
%
This research made use of
the Python language v2.7\footnote{\url{http://www.python.org/}}~\citep
{vanRossum_1995},
and the following packages:
NumPy\footnote{\url{http://www.numpy.org/}}~\citep{vanDerWalt_2011};
SciPy\footnote{\url{http://www.scipy.org/}}~\citep{Jones_2001};
Astropy\footnote{\url{http://www.astropy.org/}}, a community-developed core Python
package for Astronomy \citep{Astropy_2013};
scikit-learn\footnote{\url{http://scikit-learn.org/}}~\citep{pedregosa_2011};
matplotlib\footnote{\url{http://matplotlib.org/}}~\citep{hunter_2007}.
%
This research made use of the Tool for OPerations on Catalogues And
Tables (TOPCAT)\footnote{\url{http://www.starlink.ac.uk/topcat/}}~\citep{Taylor_2005}.
%M. Demleitner and H. Heinl (Astronomisches Rechen-Institut)
%%%%%%%%%%%%%%%%%%%%%%%%%%%%%%%%%%%%%%%%%%%%%%%%%%




%%%%%%%%%%%%%%%%%%%% REFERENCES %%%%%%%%%%%%%%%%%%

% The best way to enter references is to use BibTeX:

\bibliographystyle{mnras}
\bibliography{biblio} % if your bibtex file is called example.bib


% Alternatively you could enter them by hand, like this:
% This method is tedious and prone to error if you have lots of references
% \begin{thebibliography}{99}
% \bibitem[\protect\citeauthoryear{Author}{2012}]{Author2012}
% Author A.~N., 2013, Journal of Improbable Astronomy, 1, 1
% \bibitem[\protect\citeauthoryear{Others}{2013}]{Others2013}
% Others S., 2012, Journal of Interesting Stuff, 17, 198
% \end{thebibliography}

%%%%%%%%%%%%%%%%%%%%%%%%%%%%%%%%%%%%%%%%%%%%%%%%%%




%%%%%%%%%%%%%%%%% APPENDICES %%%%%%%%%%%%%%%%%%%%%

\appendix

\section{Color-magnitude diagrams for outliers}
\label{apdx:outliers}

We present CMDs for the ten OCs referred to as ``outliers'', given their
$\log(age/yr)>0.5$ differences between \texttt{ASteCA} and literature values.
In Fig.~\ref{fig:outliers0} the CMDs for these OCs are plotted in pairs, two OCs
per row.

The left CMD in each pair shows the cluster region with the literature isochrone
fit shown in red. For nine of the OCs we fit~\cite{Marigo_2008} isochrones,
since this set of theoretical isochrones were used in all but one of the
articles where these OCs were analysed. The exception is SMC-L35 which was fit
using~\cite{Girardi_2002} isochrones, so the same set is used for this OC.\@

The right CMD shows the same cluster region with the isochrone
that generated the best match SC found by \texttt{ASteCA}, shown in green.
The colours of the stars in this CMD correspond to the MPs assigned by the DA,
when applied.
Semi-transparent stars are those removed by the cell-by-cell density based
cleaning algorithm  (see Sect.\ref{ssec:dencontamination}), also when applied.

Values for the fundamental parameters in both fits are shown to the bottom left
of each CMD.\@

\begin{figure*}
\includegraphics[width=2.\columnwidth]{figures/outliers_VS_asteca_0.png}
\caption{CMDs for the outliers set. See description of the plots in
Appendix~\ref{apdx:outliers}.}
\label{fig:outliers0}
\end{figure*}




%%%%%%%%%%%%%%%%%%%%%%%%%%%%%%%%%%%%%%%%%%%%%%%%%%%%%%%%%%%%%%%%%%%%%%%%%%%%%%%%
%%%%%%%%%%%%%%%%%%%%%%%%%%%%%%%%%%%%%%%%%%%%%%%%%%%%%%%%%%%%%%%%%%%%%%%%%%%%%%%%

\section{Color-magnitude diagrams for P99, P00, C06, and G10 databases}
\label{apdx:databases}

We present in this appendix the CMDs of all OCs cross-matched with our own
sample, in the databases P99, P00, C06, and G10, i.e.:\ those that used the
isochrone fit method in their analysis.

\begin{figure*}
\includegraphics[width=2.\columnwidth]{figures/DB_fit/P99_VS_asteca_0.png}
\caption{CMDs for the XXX database. See description of the plots in
Appendix~\ref{apdx:databases}.}
\label{fig:DBs_P99_0}
\end{figure*}
%\clearpage

\begin{figure*}
\includegraphics[width=2.\columnwidth]{figures/DB_fit/P99_VS_asteca_1.png}
\caption{CMDs for the XXX database. See description of the plots in
Appendix~\ref{apdx:databases}.}
\label{fig:DBs_P99_1}
\end{figure*}
%\clearpage

\begin{figure*}
\includegraphics[width=2.\columnwidth]{figures/DB_fit/P00_VS_asteca_0.png}
\caption{CMDs for the XXX database. See description of the plots in
Appendix~\ref{apdx:databases}.}
\label{fig:DBs_P00_0}
\end{figure*}
%\clearpage

\begin{figure*}
\includegraphics[width=2.\columnwidth]{figures/DB_fit/P00_VS_asteca_1.png}
\caption{CMDs for the XXX database. See description of the plots in
Appendix~\ref{apdx:databases}.}
\label{fig:DBs_P00_1}
\end{figure*}
%\clearpage

\begin{figure*}
\includegraphics[width=2.\columnwidth]{figures/DB_fit/P00_VS_asteca_2.png}
\caption{CMDs for the XXX database. See description of the plots in
Appendix~\ref{apdx:databases}.}
\label{fig:DBs_P00_2}
\end{figure*}
%\clearpage

\begin{figure*}
\includegraphics[width=2.\columnwidth]{figures/DB_fit/C06_VS_asteca_0.png}
\caption{CMDs for the XXX database. See description of the plots in
Appendix~\ref{apdx:databases}.}
\label{fig:DBs_C06_0}
\end{figure*}
%\clearpage

\begin{figure*}
\includegraphics[width=2.\columnwidth]{figures/DB_fit/C06_VS_asteca_1.png}
\caption{CMDs for the XXX database. See description of the plots in
Appendix~\ref{apdx:databases}.}
\label{fig:DBs_C06_1}
\end{figure*}
%\clearpage

\begin{figure*}
\includegraphics[width=2.\columnwidth]{figures/DB_fit/C06_VS_asteca_2.png}
\caption{CMDs for the XXX database. See description of the plots in
Appendix~\ref{apdx:databases}.}
\label{fig:DBs_C06_2}
\end{figure*}
%\clearpage

\begin{figure}
\includegraphics[width=\columnwidth]{figures/DB_fit/C06_VS_asteca_3.png}
\caption{CMDs for the XXX database. See description of the plots in
Appendix~\ref{apdx:databases}.}
\label{fig:DBs_C06_3}
\end{figure}
%\clearpage

\begin{figure*}
\includegraphics[width=2.\columnwidth]{figures/DB_fit/G10_VS_asteca_0.png}
\caption{CMDs for the XXX database. See description of the plots in
Appendix~\ref{apdx:databases}.}
\label{fig:DBs_G10_0}
\end{figure*}
%\clearpage

\begin{figure*}
\includegraphics[width=2.\columnwidth]{figures/DB_fit/G10_VS_asteca_1.png}
\caption{CMDs for the XXX database. See description of the plots in
Appendix~\ref{apdx:databases}.}
\label{fig:DBs_G10_1}
\end{figure*}
%\clearpage

\begin{figure*}
\includegraphics[width=2.\columnwidth]{figures/DB_fit/G10_VS_asteca_2.png}
\caption{CMDs for the XXX database. See description of the plots in
Appendix~\ref{apdx:databases}.}
\label{fig:DBs_G10_2}
\end{figure*}
%\clearpage

\begin{figure*}
\includegraphics[width=2.\columnwidth]{figures/DB_fit/G10_VS_asteca_3.png}
\caption{CMDs for the XXX database. See description of the plots in
Appendix~\ref{apdx:databases}.}
\label{fig:DBs_G10_3}
\end{figure*}
%\clearpage

\begin{figure*}
\includegraphics[width=2.\columnwidth]{figures/DB_fit/G10_VS_asteca_4.png}
\caption{CMDs for the XXX database. See description of the plots in
Appendix~\ref{apdx:databases}.}
\label{fig:DBs_G10_4}
\end{figure*}
%\clearpage

\begin{figure*}
\includegraphics[width=2.\columnwidth]{figures/DB_fit/G10_VS_asteca_5.png}
\caption{CMDs for the XXX database. See description of the plots in
Appendix~\ref{apdx:databases}.}
\label{fig:DBs_G10_5}
\end{figure*}
%\clearpage

\begin{figure*}
\includegraphics[width=2.\columnwidth]{figures/DB_fit/G10_VS_asteca_6.png}
\caption{CMDs for the XXX database. See description of the plots in
Appendix~\ref{apdx:databases}.}
\label{fig:DBs_G10_6}
\end{figure*}
%\clearpage

\begin{figure*}
\includegraphics[width=2.\columnwidth]{figures/DB_fit/G10_VS_asteca_7.png}
\caption{CMDs for the XXX database. See description of the plots in
Appendix~\ref{apdx:databases}.}
\label{fig:DBs_G10_7}
\end{figure*}
%\clearpage

\begin{figure}
\includegraphics[width=\columnwidth]{figures/DB_fit/G10_VS_asteca_8.png}
\caption{CMDs for the XXX database. See description of the plots in
Appendix~\ref{apdx:databases}.}
\label{fig:DBs_G10_8}
\end{figure}




%%%%%%%%%%%%%%%%%%%%%%%%%%%%%%%%%%%%%%%%%%%%%%%%%%%%%%%%%%%%%%%%%%%%%%%%%%%%%%%%
%%%%%%%%%%%%%%%%%%%%%%%%%%%%%%%%%%%%%%%%%%%%%%%%%%%%%%%%%%%%%%%%%%%%%%%%%%%%%%%%

\section{Color-magnitude diagrams for large mass OCs}
\label{apdx:largemass}

We present here CMDs for the five OCs with the largest mass estimates by the
either the H03 or the P12 database. These are also the OCs with the largest
differences in assigned masses when compared with those derived by
\texttt{ASteCA}.
See Table~\ref{tab:integ_phot_masses} for more information about the fundamental
parameters assigned to these OCs.

\begin{figure*}
\includegraphics[width=2.\columnwidth]{figures/largemass_VS_asteca_0.png}
\caption{CMDs for the OCs with the largest masses assigned by H03 and/or P12.
The best match synthetic cluster is plotted to the right, and the observed
cluster region CMD to the left, for each OC.}
\label{fig:largemass}
\end{figure*}




%%%%%%%%%%%%%%%%%%%%%%%%%%%%%%%%%%%%%%%%%%%%%%%%%%%%%%%%%%%%%%%%%%%%%%%%%%%%%%%%
%%%%%%%%%%%%%%%%%%%%%%%%%%%%%%%%%%%%%%%%%%%%%%%%%%%%%%%%%%%%%%%%%%%%%%%%%%%%%%%%

\section{Age-metallicity relations for literature values}
\label{apdx:amr_lit}

In Fig.~\ref{fig:largemet} the CMDs for the four LMC OCs with metallicities and
ages in the vicinity of ${\sim}3.3$ Gyr and [Fe/H]${\approx-0.25}$ dex, are
shown. These are the OCs responsible for the high metallicity found for the AMR
of this galaxy, for ages close to 4 Gyr.

\begin{figure*}
\includegraphics[width=2.\columnwidth]{figures/largemet_VS_asteca_0.png}
\caption{CMDs for the OCs that cause a peak in the AMR for the LMC, for ages
close to 4 Gyr.
The best match synthetic cluster is plotted to the right, and the observed
cluster region CMD to the left, for each OC.}
\label{fig:largemet}
\end{figure*}

Fig.~\ref{fig:amr_lit} shows the AMRs for both Clouds, generated using the
metallicity and age values taken from those articles presented in
Table.~\ref{tab:literature}.

\begin{figure}
\includegraphics[width=\columnwidth]{figures/AMR_literature.png}
\caption{XXX}
\label{fig:amr_lit}
\end{figure}

%%%%%%%%%%%%%%%%%%%%%%%%%%%%%%%%%%%%%%%%%%%%%%%%%%


% Don't change these lines
\bsp	% typesetting comment
\label{lastpage}
\end{document}

% End of mnras_template.tex

% % Example figure
% \begin{figure}
% 	% To include a figure from a file named example.*
% 	% Allowable file formats are eps or ps if compiling using latex
% 	% or pdf, png, jpg if compiling using pdflatex
% 	\includegraphics[width=\columnwidth]{example}
%     \caption{This is an example figure. Captions appear below each figure.
% 	Give enough detail for the reader to understand what they're looking at,
% 	but leave detailed discussion to the main body of the text.}
%     \label{fig:example_figure}
% \end{figure}

% % Example table
% \begin{table}
% 	\centreing
% 	\caption{This is an example table. Captions appear above each table.
% 	Remember to define the quantities, symbols and units used.}
% 	\label{tab:example_table}
% 	\begin{tabular}{lccr} % four columns, alignment for each
% 		\hline
% 		A & B & C & D\\
% 		\hline
% 		1 & 2 & 3 & 4\\
% 		2 & 4 & 6 & 8\\
% 		3 & 5 & 7 & 9\\
% 		\hline
% 	\end{tabular}
% \end{table}

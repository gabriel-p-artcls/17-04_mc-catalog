\documentclass{article}
\usepackage{graphicx}
\usepackage{hyperref}
\usepackage{natbib,twoopt}
\usepackage{microtype}
\usepackage{afterpage}

\usepackage{xcolor}
\newcommand{\finish}{\textcolor{red}{\textbf{$\;\;$|-- FINISH --|$\;\;$}}}

\begin{document}

\title{Referee's comments and suggestions}
\author{GI Perren, AE Piatti, RA V\'azquez}

\maketitle

\renewcommand\thesubsection{\Alph{subsection}}
\begin{abstract}
We thank the referee for their detailed revision of our article.

Below are the responses to the referee's comments and suggestions, following
their numbering.
\end{abstract}

\section*{General concerns}

\subsection{}
\label{gen_a}

As the editor points out, this was done by their office to shrink the size
of the PDF.\@ All the images used in the manuscript are of high quality.

For the referee and the editor's convenience, we make all images available in
the following link at their full size:
\url{https://www.dropbox.com/sh/jlrwttw2220adfp/AABwqBNNt7GzcID0CKi6si8Va?dl=0}

This url will be accessible until the reviewing process finishes.

\subsection{}
The new version of the manuscript was shortened as shown in the following
table:

\begin{tabular}{ l c r }
& Original & New version \\
Words in text & 16503 & 13793 \\\\ % 13203~ 20%
\end{tabular}

The 2713 words difference represents a ${\sim16}$\% reduction, and is the
most text we believe we can remove without affecting the content substantially.


\clearpage
\renewcommand\thesubsection{\arabic{subsection}}
\section*{Comments and questions}
\setcounter{subsection}{0}

% 1
\subsection{}
The sentence was changed to improve its readability.

% 2
\subsection{}
The referee is correct, there are many articles where the analysis of star
clusters is not done by-eye. In Sect 2.9 of Paper I~\citep{Perren_2015} we
mention approximately thirty of these articles. It was certainly not our
intention to convey that all studies but this one are done by-eye.

A few sentences were added to make this more clear.

% 3
\subsection{}
We added a sentence to explain why these clusters were used.

% 4
\subsection{}
We corrected that all along the manuscript.

% 5
\subsection{}
Corrected.

% 6
\subsection{}
The maximum mass was selected after a first rough estimate of the masses of the
clusters in our dataset (a first pass with ASteCA), using
$1\times10^4\;M_{\odot}$ as the limit. In all but a handful of cases, the masses
where estimated below $5\times10^3\;M_{\odot}$; making the
$1\times10^4\;M_{\odot}$ Mo max limit a reasonable value. For the 15 clusters
which showed total masses close to $1\times10^4\;M_{\odot}$, we extended this
limit to $3\times10^4\;M_{\odot}$.

These details were added to the second to last paragraph of Sect 3.1.

\subsubsection{Add}

1.
The $3\times10^4\;M_{\odot}$ maximum value proved to be a very
reasonable limit for the most massive clusters, as shown later in Sect 5.2.1.
There, we demonstrate that even if no field star cleaning is performed (i.e., we
use all stars in the cluster region when searching for the best synthetic
cluster match), and using a max mass limit of $5\times10^5\;M_{\odot}$, the
largest cluster (NGC419) is assigned a total mass of $5.5\times10^4\;M_{\odot}$.
We know that this is an overestimated value, since we fitted not only the
cluster stars but all contaminating field stars too, which shows that
$3\times10^4\;M_{\odot}$ is a reasonable limit.\\

\noindent 2.
The step defines the density of the selected parameter range, and it is fixed
throughout the best match process. The value of the step
is defined by the user along with the min and max values of the parameter's
range (in the \texttt{params\_input.dat} file). This is true not only for the
mass but for all the fitted parameters (age, metallicity, etc.)
Picking a larger step would only decrease the accuracy of the parameter
estimate, and it would have very little (if not no) impact on the final
selected mass value (or any other parameter).

% 7
\subsection{}
First: we added a portion of the observed cluster region above the image, as
requested. The caption was modified to reflect this change, and to make the
description of the overall plot clearer.\\

\noindent Second: that must be related to the compression performed by the
editorial office. Please see the corresponding image shared in the url presented
in Sect.~\ref{gen_a}.

% 8
\subsection{}
\label{sec:8}
The limiting magnitude is taken individually for each observed/processed
cluster. It is the limiting magnitude of the observed cluster itself, i.e., it
varies with each observed cluster.
This is explained in Paper I, Sect 2.9.1. At the beginning of Sect 3.4 
(footnote 17) we sate that the details of the code's built-in functions are
explained in Paper I. We did not repeat them here to avoid redundancy.

We added a sentence in the caption of Fig. 2 explaining this.

The name of this section was modified to \emph{Synthetic CMD matching}, to
better reflect its content.

% 9
\subsection{}
As explained in Paper I Sect 2.9.2, we employ a genetic algorithm (GA) to
find the best synthetic cluster match for each observed cluster. The GA explores
the entire parameters space, which is composed -- in this case -- of
approximately $2.3\times10^7$ possible models/solutions (i.e: synthetic CMDs),
as shown in Table 3.
In this work, each observed cluster was compared by the GA to roughly
$1.5\times10^6$ models (including those explored during the bootstrap
runs). For each of these models, a unique random IMF sampling is
used to populate its isochrone. 

We added a sentence in this paragraph, and at the end of the section explaining
this.

% 10
\subsection{}
Changed ``usually'' to ``often'' as suggested.

% 11
\subsection{}
Changed ``versus'' to ``of'' as suggested.

% 12
\subsection{}
The fonts of labels and values In fig. 3 have been increased, as requested.

% 13
\subsection{}
We use $z$ instead of [Fe/H] for two reasons.
The first reason is that this is the default form in which the evolutionary
tracks are downloaded from by Girardi's CMD service 
(\url{http://stev.oapd.inaf.it/cgi-bin/cmd}).
In the near future, we would like to add other sets of tracks (Darmouth,
Geneva, BaSTI, MIST-MESA, etc).
The second reason is that it is easier to develop the code when all
fundamental parameters are strictly positive.

\begin{figure}
    \centering
    \includegraphics[width=5in]{figs/errors_z.png}
    \caption{Error in $z$.}
    \label{fig:errors_z}
\end{figure}

The errors are displayed using [Fe/H] to make the reader aware of the issue
that [Fe/H] errors will increase for low metallicities, for a mathematical
reason.
If we had plotted the errors using $z$ instead, we'd get a diagram that is not
much informative, as the errors in $z$ are (as stated in the article) very
clustered around the $\sigma_{z}{=}0.003$ value. This can be seen in
Fig.~\ref{fig:errors_z}.

A footnote was added in paragraph 3 of Section 4, to address this concern.

% Furthermore, if the code used [Fe/H] values we would still have to obtain them
% from the Z values with which the evolutionary tracks are created. This forces
% the choice of a solar Z, and since this value was updated from the canonical
% $Z_o=0.019$ to $Z_o=0.152$ for the latest Bressan isochrones, we'd still encounter
% the issue of 

% 14
\subsection{}

We meant ``low'' compared to solar; perhaps this wasn't the best choice of
words. The ``confirmation bias'' effect explains that researchers will tend
to select values that have already been used, which explains the clustering of 
[Fe/H] around the -0.4 dex and -0.7 dex values in the literature.

We've re-written that paragraph adding a more detailed explanation of the
effect.

% 15
\subsection{}
The referee is correct, we did not mean to imply that there was not a good
reason for doing this.

We've added this information to the paragraph.

% 16
\subsection{}
We are not saying that the distance moduli is fixed in all studies. We
are referring to the ``literature'' articles (the 19 shown in Table 1), were the
distance moduli were indeed always assumed to be fixed.
That's what we meant by saying ``in the latter'', referring to the previous
sentence which mentioned these literature articles.

We have changed the wording to explicitly mention the ``literature articles''.


% 17
\subsection{}
Corrected here, and also in the second to last paragraph of Sect 5.2 where the
same name for the plot was used.

% 18
\subsection{}
\label{sec:18}
\textbf{On the determination of cluster masses}\\

The mass estimation done by \texttt{ASteCA} via the GA relies on the same exact
method used to estimate the other fundamental parameters (metallicity, age,
etc.) This is explained in Paper I, Sect. 2.9.1 and 2.9.2. The only thing that
changes from Paper I to this one, is the likelihood function.
The~\cite{Dolphin_2002} likelihood is used here, because it allows the
estimation of mass through the comparison of the number of stars in cells (2D
bins) of the CMD's histogram. Basically, this amounts to a comparison between
the ``Hess diagrams'' of an observed cluster versus that of the generated
synthetic cluster.

When a synthetic CMD of a given metallicity and age is created via a theoretical
isochrone (in our case, we used the~\cite{Bressan_2012} set), the entire
theoretical isochrone is taken into account, down to the faintest magnitude.
%
Given a total mass value, this isochrone is populated through sampling of the
selected IMF (in the case of this work, we used the~\cite{Chabrier_2001}
log-normal IMF, as mentioned in Sect. 3.4) until that fixed total initial mass
value is achieved. This is: the synthetic CMD is generated accounting for all
stars in the theoretical isochrone.

After this, the synthetic CMD is perturbed by the same processes that
affect the observed cluster, i.e., limiting observed magnitude, incompleteness
of observed stars, and photometric errors. This is done to
make the synthetic CMD comparable to the observed one.
%
The two first processes will effectively remove stars from the synthetic CMD,
particularly the limiting magnitude effect. These stars are removed because
there is nothing to compare them to, since the observed cluster does not
contain stars beyond its observed magnitude limit.
%
% Even though these stars are removed prior to the fitting of a synthetic CMD 
% (with the observed CMD), the mass associated with this cluster remains the
% total mass used to create it.

For example: a synthetic CMD of $10000\,M_{\odot}$ is generated, and after the
removal of faint stars beyond the magnitude limit its mass is ``reduced'' to
$2000\,M_{\odot}$. I.e., the mass of the stars that are visible in the synthetic
CMD, after the magnitude limit removal, is $2000\,M_{\odot}$.
If this synthetic CMD results the one with the best match to the observed
cluster (i.e., its likelihood value is the maximum found for all the synthetic
CMDs processed), the estimated mass assigned to the observed cluster will be
$10000\,M_{\odot}$, not $2000\,M_{\odot}$, because that is the actual
mass that was used to generate the synthetic CMD.\@

To be clear, at no point during our analysis do we make use of either
integrated magnitudes or M/L relations to estimate masses. The mass estimation
depends exclusively on the maximization of the PLR.\@
There is no extrapolation performed since none is needed: all stars are
taken into account at the moment of the synthetic CMD's generation.
Fig. 8 in Paper I shows the full process of generating a synthetic CMD.\\\\

\noindent \textbf{On the mass fraction outside the radius}\\

We are not entirely sure what the referee is referring to when they mention the
``fraction of the cluster's mass that is located outside the cluster radius''.
Are they talking about the dynamical effect of mass loss due to evaporation or
two-body relaxation?
If this is the case, given that mass loss tends to affect low-mass
stars preferentially~\citep{Vesperini_2010}, this issue will have a minor impact
on our mass determination. This is because we asses the total cluster mass by
directly comparing the visible portion of the cluster's CMD with a large number
of synthetic CMDs (after affecting these synthetic CMDs with similar processes
to  that which affect the observed cluster, like the limiting magnitude for
example). Since the portion of lowest mass stars is not observed (due to the
limiting magnitude), it will not affect the matching process.

This shouldn't be confused with the assumption that low-mass stars (down to
0.01 $M_{\odot}$) are not considered at all in the mass estimation process. They
are indeed taken into account at the moment of the synthetic CMD's generation,
when the fixed total mass of the synthetic cluster is distributed along its
theoretical isochrone via IMF sampling.\\

If the referee is concerned about luminosity being lost outside a certain
measurement aperture, and thus affecting the mass estimation, this is not an
issue for \texttt{ASteCA}. The observed cluster's mass is obtained (as stated
in the subsection above) without making use of integrated magnitudes or M/L
relationships. The code relies entirely on the comparison of the distribution
and number of stars in the observed CMD, with that of many synthetic CMDs.\\

If the question was aimed at the more simple issue of some stars being left
out of the selected cluster region, we do not believe this affects the estimated
total mass.
Because we employ the radius of the cluster as the limit where the star density
falls to that of the field, we do not expect to be missing any significant
portion of the observed cluster.\\

We added a couple of sentences to the first, second, and third paragraphs of
Sect.~3.4 in the new version of the article, which should address the doubts
regarding total mass estimation.


% 19
\subsection{}
\label{sec:19}
\textbf{On the M/L relation of the H03 and P12 articles}\\

If we understood correctly, the referee is asking us to calculate a
normalized mass via a relation of the form

\begin{equation}
M_{P12,norm} = M_{P12} \frac{(M/L)_{H03}}{(M/L)_{P12}} =
\frac{M_{H03} L_{P12}}{L_{H03}}
\label{eq:mass_norm}
\end{equation}

\noindent where $M_{H03}$, $(M/L)_{H03}$, $M_{P12}$, $(M/L)_{P12}$ are the
masses and mass-luminosity relations of~\citet[][H03]{Hunter_2003}
and~\citet[][P12]{Popescu_2012} respectively, and
$M_{P12,norm}$ is the mass of P12 normalized to the $(M/L)_{H03}$ relation.

If this is the request, we must point out that P12 used the same database of
integrated magnitudes as H03, which means that the luminosities used for each
cluster are equal ($L_{P12}=L_{H03}$). This would reduce Eq.~\ref{eq:mass_norm}
to

\begin{equation}
M_{P12,norm} = M_{H03}
\end{equation}

\noindent meaning the mass differences would be zero for all clusters. If this
is not what the referee pointed out, we ask them to please expand a bit
on their request.\\

It is also important to point out that P12 did not employ a standard
mass-luminosity relationship to estimate masses. Rather, the authors used
their MASSCLEAN$age$ package~\citep[which in turn depends on their
MASSCLEAN$colors$
package,][]{Popescu_2010b,Popescu_2010a}. This tool allows the simultaneous
determination of the most probable age-mass values, from the observed integrated
magnitudes matched in a hyper-space of synthetic integrated magnitudes.
%
These articles by Popescu et al.~also highlight the enormous degenerations in
the solutions that arise, when simple integrated magnitude methods are used for
mass and age estimations.\\\\

\noindent \textbf{On the mass discrepancy of NGC 419, 1917, 1751}\\

We thank the referee for taking the time to track down and calculate
the mass values presented in this section, for these three clusters.

Obtaining mass and age estimates from integrated magnitudes is known to be a
process affected by large stochasticity (see for example the Popescu \& Hanson
articles). Below we show diagrams for each of these three clusters, where we
explain in detail how different mass values affect the generated synthetic CMD..
This should help the referee understand why we believe our mass estimates are
well-estimated, and the large mass values given in the integrated magnitude
articles are overestimated.\\

\noindent \emph{* \textbf{NGC 1917}}, Fig.~\ref{fig:ngc1917}\\

\noindent - Panel \textbf{(a)}: CMD of the observed cluster region. In our
original analysis we used a radius of 27.4 arcsec which resulted in the 4000
$M_{\odot}$ total mass estimation. Here, we use instead a larger radius of 62
arcsec, equal to the aperture size used by~\cite{vandenBergh_1981}.
%
$N_{accpt}$ is the number of stars that were not rejected due to
large photometric errors, and $n_{memb}$ is the number of approximated cluster
members. This last value is obtained by averaging the number of stars in
the surrounding field (next panel) and subtracting that value from the total
number of stars within the cluster region. The final result is ${\sim}169$
expected cluster members.

\noindent - Panel \textbf{(b)}: CMD of ten combined surrounding field
regions, each with an area equal to that of the cluster.

\noindent - Panel \textbf{(c)}: CMD of the cluster region after the
decontamination algorithm (DA) was applied. Stars are colored according to the
membership probabilities (MPS) assigned, semi-transparent stars are those
removed by the cell-by-cell density based decontamination process.
$N_{fit}$ is the number of stars that are left after this removal.
Ideally, $n_{memb}$ and $N_{fit}$ should be similar, since they are both rough
(independent) estimates of the (visible) number of cluster members.

\noindent - Panel \textbf{(d)}: best match synthetic cluster found using the
larger radius of 62 arcsec.
Its fundamental parameter values are fixed to $z{=}0.008$,
$\log(age){=}9.15$, $(m-M){=}18.48$, and $E_{(B-V)}{=}0.08$, as obtained
originally by \texttt{ASteCA}. The mass is allowed to vary in a range between 
[500, $1{\times}10^5$] $M_{\odot}$.
%
$N_{synth}$ is the number of visible stars in this best match synthetic CMD 
(notice that this value is rather close to $N_{fit}$, the approximated number of
cluster members), and $M{=}1.1{\times}10^4\,M_{\odot}$ is the total mass
used to generate this synthetic CMD (not to be confused with the
portion of mass visible in the CMD, which is much smaller).
We see that this mass is larger than the 4000 $M_{\odot}$ originally estimated.
This is because the used radius is now more than twice as large, and because for
relatively low mass clusters the stochasticity in their mass assignment is
non-negligible.

\noindent - Panel \textbf{(e)}: best match synthetic CMD using the same
fundamental parameter values and mass range shown in \textbf{(d)}, where the
full cluster region with no decontamination procedure applied was used.
I.e., we fit here the $\sim1700$ stars in the observed cluster region, which
will obviously include a majority of field stars.
The best match estimated mass increases to $4.9{\times}10^4\,M_{\odot}$.
This shows that even if no field star cleaning is performed, the estimated mass
would still not be close to the mass given in articles where integrated
magnitudes were used.

\noindent - Panel \textbf{(f)}: synthetic CMD generated using the same
fundamental parameters mentioned in \textbf{d}, and a fixed mass of
$8{\times}10^4\,M_{\odot}$ (average mass value estimated by H03, P12, and the
referee's own calculations).
The number of stars in this CMD is more than two times larger than the stars
present in the CMD of the observed cluster region, even if all
contaminating field stars are taken into account.
It is visible at plain sight, with no statistical estimator needed, that the
large mass value given in other articles produces a number of stars in the best
match synthetic CMD, that is not compatible with the observed
cluster.\\

% NGC1917 (0.274 arcsec/px)
%
% ASteCA:
% 100 px -> 27.4 arcsec; log(age)=9.15, dm=18.48, E_BV=0.08, z=0.002 --> M=4e3
% Mo
% van den Bergh (1981, A&AS, 46, 79):
% 62 arcsec --> r_cl = 226 px; log(age)=9.15, dm=18.48, z=0.008 --> M=8e4 Mo
% H03: M=5.9e4 Mo  ; P12: M=1e5 Mo ; average: 7.95e4 Mo
%

\afterpage{%
\thispagestyle{empty}
\begin{figure}[p]
    \vspace*{-3.7cm}
    \makebox[\linewidth]{
        \includegraphics[width=1.3\linewidth]{figs/NGC1917_mass.png}
    }
    \caption{Analysis of NGC1917.}
\label{fig:ngc1917}
\end{figure}
\clearpage
}

\noindent \emph{* \textbf{NGC 419}}, Fig.~\ref{fig:ngc419}\\

\noindent - Panels \textbf{(a)}, \textbf{(b)}, and \textbf{(c)} are equivalent
to those from Fig.~\ref{fig:ngc1917}.
The radius used in~\cite{Goudfrooij_2014} to derive their mass estimate of
${\sim}2.2{\times}10^5\,M_{\odot}$, is 50 arcsec. In our original analysis of
this cluster we employed a larger radius of 85 arcsec, which is also
used here.

\noindent - Panel \textbf{(d)}: original mass estimate, where we see
that the number of stars in the synthetic CMD is rather close to the estimated
number of cluster members.

\noindent - Panel \textbf{(e)}: best match estimated mass using
the full cluster region, with no field star cleaning performed. The fundamental
parameters were fixed to $z{=}0.012$, $\log(age){=}8.95$, $(m-M){=}18.92$, and
$E_{(B-V)}{=}0.02$, as obtained originally by \texttt{ASteCA}. The mass is
allowed to vary in a range between [500, $1{\times}10^5$] $M_{\odot}$. The best
match mass is twice as large as the mass value obtained using the decontaminated
cluster region.

\noindent - Panel \textbf{(f)}: how the CMD should look if
this cluster had a mass of ${\sim}2.7{\times}10^5\,M_{\odot}$, as estimated on
average by H03, and Goudfrooij et al.
Again, we see that the number of stars in this large mass synthetic CMD, is
not compatible with the observed cluster.\\

% NGC419 (0.274 arcsec/px)
%
% AsteCA:
% 310 px -> 85 arcsec; log(age)=8.95, dm=18.92, E_BV=0.02, z=0.012 --> M=2.8e4
% Goudfrooij et al. (2014, ApJ, 797, 35):
% 50 arcsec -> 182 px; log(age)=9.16, dm=18.85, E_BV=0.05, z=0.004 -->
% M=2.4e5 (Salpeter), 1.9e5 (Chabrier)
% H03: M=3.9e5 Mo; average (Goudfrooij, H03): 2.7e5 Mo
%

\afterpage{%
\thispagestyle{empty}
\begin{figure}[p]
    \vspace*{-3.7cm}
    \makebox[\linewidth]{
        \includegraphics[width=1.3\linewidth]{figs/NGC419_mass.png}
    }
    \caption{Analysis of NGC419.}
\label{fig:ngc419}
\end{figure}
\clearpage
}

\noindent \emph{* \textbf{NGC 1751}}, Fig.~\ref{fig:ngc1751}\\

\noindent - Panels \textbf{(a)}, \textbf{(b)}, and \textbf{(c)} are equivalent
to those from Fig.~\ref{fig:ngc1917}.
The radius used in~\cite{Goudfrooij_2014} is 50 arcsec. We use here a radius of
60 arcsec, the same employed in our original analysis of this cluster.

\noindent - Panel \textbf{(d)}: original mass estimate.

\noindent - Panel \textbf{(e)}: best match estimated mass using the full cluster
region, with no field star cleaning performed. The fundamental
parameters were fixed to $z{=}0.012$, $\log(age){=}9.1$, $(m-M){=}18.6$,
and $E_{(B-V)}{=}0.04$, as obtained originally by \texttt{ASteCA}. The mass is
allowed to vary in a range between [500, $1{\times}10^5$] $M_{\odot}$. The best
match mass is four times the mass value obtained using the decontaminated
cluster region.

\noindent - Panel \textbf{(f)}: how the CMD should look if
this cluster had a mass of ${\sim}7.2{\times}10^4\,M_{\odot}$, as estimated on
average by P12, H03, and Goudfrooij et al.
Once more, the number of stars in this large mass synthetic CMD, is
not compatible with the observed cluster.\\

% NGC1751 (0.274 arcsec/px)
%
% ASteca:
% 220 px -> 60 arcsec; log(age)=9.1, dm=18.6, E_BV=0.04, z=0.012 --> M=9e3
% Goudfrooij et al. (2014, ApJ, 797, 35):
% 50 arcsec -> 182 px; log(age)=9.15, dm=18.5, E_BV=0.12, z=0.008 -->
% M=6.5e4 (Salpeter), 6e4 (Chabrier)
% H03: M=9.7e4; P12: 6.5e4; average (Goudfrooij, H03, P12): 7.2e4 Mo
%

\afterpage{%
\thispagestyle{empty}
\begin{figure}[p]
    \vspace*{-3.7cm}
    \makebox[\linewidth]{
        \includegraphics[width=1.3\linewidth]{figs/NGC1751_mass.png}
    }
    \caption{Analysis of NGC1751.}
\label{fig:ngc1751}
\end{figure}
\clearpage
}

\clearpage
\noindent \emph{* \textbf{MASSCLEAN synthetic CMDs}}\\

The referee could of course question the ability of \texttt{ASteCA} in producing
these synthetic CMDs, of given metallicities, ages, and masses.
%
To further strengthen our analysis, we present in
Fig.~\ref{fig:massclean_mass} three synthetic CMDs generated with the
MASSCLEAN tool~\citep[][]{Popescu_2009}.

These CMDs were created using the same fundamental parameters assigned to NGC
1917, 419, and 1751 by \texttt{ASteCA}, but using an averaged total mass
obtained from the values given to each of them by H03, P12, and the articles
mentioned by the referee.
This means that these CMDs are the equivalent of those in panels
\textbf{(f)} of Figs.~\ref{fig:ngc1917},~\ref{fig:ngc419},
and~\ref{fig:ngc1751}, but produced with a completely independent tool.

MASSCLEAN does not support the Washington photometric system, hence the $V$
versus $(B-V)$ diagrams. These synthetic CMDs are affected by a comparable
limiting magnitude (basically $V_{lim}=T1_{lim} + 0.5$ mag) and incompleteness,
as those affecting the observed CMDS for each of the three clusters. The
differences with the synthetic CMDs produced by \texttt{ASteCA} are due to the
stochastic nature of IMF sampling (MASSCLEAN uses a Kroupa-Salpeter IMF, we
used the~\citealp{Chabrier_2001} log-normal IMF), the theoretical isochrones
set (MASSCLEAN uses the old~\citealp{Marigo_2008} set, while our code employs
the newer~\citealp{Bressan_2012} set), and binarity (MASSCLEAN does not support
the addition of binaries).

Again, we can clearly see that these synthetic CMDs do not represent
the CMDs of the observed clusters. If these observed clusters had the large
masses estimated by other works, the number of stars that should be
present in their CMDs would need to be much larger than the number we
actually observe.
On average, we should observe between 120\%-570\% more stars if
we compared with the entire cluster region (no decontamination process applied),
and between 1200\%-1900\% if we compare with the cluster region after
removing the contaminating field stars.\\

\begin{figure}
    \includegraphics[width=\linewidth]{figs/massclean_mass.png}
    \caption{Synthetic CMDs for NGC 1917, 419, and 1751 produced by the
    MASSCLEAN package, using their large mass estimates.}
\label{fig:massclean_mass}
\end{figure}


\noindent \emph{* \textbf{Concluding remarks}}\\

The conclusion that can be drawn from this analysis is thus equivalent to that
presented in the article: these Magellanic clusters have had their masses
overestimated by a substantial amount in the H03 and P12
works~\citep[and others too, i.e. the][article mentioned by the referee]
{Goudfrooij_2014}.
The link between these studies is the use of integrated magnitudes to derive
masses.

Finally, to reinforce this notion we refer the referee to Appendix A of the
original article.
The validation there performed shows that \texttt{ASteCA} recovers
masses with very reasonable accuracy. Furthermore, the masses are recovered with
increased accuracy, as the clusters grow larger. Having used an external
package (MASSCLEAN) to generate the almost 800 synthetic clusters used in this
validation, there is no possibility for internal biases in the mass
determination.
Again, this leads to the same conclusion: \texttt{AsteCA} is performing a proper
estimation of masses within its uncertainties, and the cause of the discrepancy
between the masses must be located in those articles where the classical method
of mass estimation via integrated magnitudes is employed.


% 20
\subsection{}
Corrected, as suggested by the referee.

% 21
\subsection{}
Corrected, as suggested by the referee.

% 22
\subsection{}
As the luminosity of a star cluster decreases with age, the oldest clusters that
can be detected are those with the largest masses (and hence larger radii)
This is discussed for example in~\citet[][Sect. 4]{Popescu_2012}.
Fig.~\ref{fig:popescu_2012} shows Fig. 16 from that article, where a fading
limit is shown (black line) and the LMC clusters can be seen to increase in mass
as their ages increase.
We've added a reference to this effect and the above mentioned article.

\begin{figure}
    \centering
    \includegraphics[width=5in]{figs/popescu_2012.png}
    \caption{Figure taken from Popescu et al. (2012, Sect. 4)}
\label{fig:popescu_2012}
\end{figure}

% 23
\subsection{}
The referee is right to point this out (this discussion was actually in the
first draft of our article, but we cut it to save space).
If this group of three ``high'' metallicity and ``old'' age ($\log(age)\ge9.5$)
LMC clusters is removed, the peak disappears and hence, so
does the drop. In Fig.~\ref{fig:amr_no_high_met} we show how the LMC's AMR
would look without these three clusters present.
We've added a couple of sentences in the corresponding paragraph of the article,
addressing this effect.

\begin{figure}
    \centering
    \includegraphics[width=4in]{figs/AMR_asteca_no_lmc_high_met.png}
    \caption{\texttt{ASteCA} AMR without the three old LMC clusters with
    high metallicities values.}
\label{fig:amr_no_high_met}
\end{figure}

% 24
\subsection{}
We believe we have addressed the issues brought up by the referee in this cover
letter, particularly in Sects.~\ref{sec:18} and~\ref{sec:19}, regarding the mass
assignment performed by the code.
As such, seeing no reason to alter them, we stand by our original conclusions
which we soften changing ``are shown'' to ``appear'', and ``moderate'' to
``better''.

%-------------------------------------------------------------------
\bibliographystyle{aa}
\bibliography{biblio} % your references Yourfile.bib

\end{document}
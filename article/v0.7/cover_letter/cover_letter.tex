\documentclass{article}
\usepackage{graphicx}
\usepackage{hyperref}
\usepackage{natbib,twoopt}
\usepackage{microtype}
\usepackage{afterpage}
\usepackage{amsmath}

\usepackage{xcolor}
\newcommand{\finish}{\textcolor{red}{\textbf{$\;\;$|-- FINISH --|$\;\;$}}}

\begin{document}

\title{Referee's comments and suggestions (Revision 2)}
\author{GI Perren, AE Piatti, RA V\'azquez}

\maketitle

\begin{abstract}
We thank the referee for this second detailed revision of our article.

Below are the responses to the referee's comments and suggestions, following
their numbering.
\end{abstract}

\clearpage
\renewcommand\thesubsection{\arabic{subsection}}
% \section*{Comments and questions}

% 9
\setcounter{subsection}{8}
\subsection{Sect. 3.4, paragraph 2}
With respect to the weight given in the fit process, the likelihood used in this
work treats all bins in the same manner. This is, no ``strong weight'' is given
to the post-MS region, or any other region of the CMD.\\

\noindent For low mass clusters the stochastic effects will indeed be much more
noticeable. This reflects in the fact that those clusters with the lowest mass
estimates, are generally those which present more issues in their matching.
These issues are mentioned throughout the article a number of times:\\

\noindent * \textbf{Sect. 5}: ``We find in Appendix A that the code will
slightly underestimate the masses by approximately 200 $M_{\odot}$, for low mass
clusters.''\\
\noindent * \textbf{Sect. 5.2.1}: ``The standard deviation of the average
relative mass differences decreases for larger masses (see Fig. 9) as expected,
pointing to a more accurate recovery of the true total mass by \texttt{ASteCA},
as the clusters grow in size.''\\
\noindent * \textbf{Appendix A}\@: ``For low mass clusters -- $500\,M_{\odot}$
or $1000\,M_{\odot}$ -- the code assigns masses in a range between ${\sim}200
{-}3000$ $M_{\odot}$. In this region \texttt{ASteCA} underestimates clusters'
masses by ${\sim}200\,M_{\odot}$. This effect is tied to an improper age
estimation, where \texttt{ASteCA} incorrectly assigns younger ages to scarcely
populated clusters, and compensates the low number of stars by decreasing the
total mass. Such an issue is not unexpected for very low mass clusters.''\\
\noindent * \textbf{Appendix A.1}: ``Most of the poorest solutions obtained by
\texttt{ASteCA} -- those with $|\Delta z|{>}0.01$ dex -- are associated to low
mass scarcely populated clusters, with ${\sim}40$ true member stars on average 
(from two up to a hundred) in their analyzed CMDs. This poor solutions set is
composed of 91 clusters -- ${\sim}12\%$ of the sample -- 58 of which have
$M{\le}1000\,M_{\odot}$.
Of these 58 low mass clusters, 38 are assigned younger ages by the code due to
an improper field star decontamination process (an expected issue when the
number of true members is very low).''\\

\texttt{ASteCA} does not currently provide a ``goodness of fit''
parameter. The best fit obtained by the code is relative to all the
remaining synthetic CMDs analyzed (hundreds of thousands, or millions), but
there's no value to quantify how good this fit is. The original
article where the employed likelihood is derived~\citep{Dolphin_2002}, gives a
method to asses this goodness of fit (Sect. 2.6), but it isn't yet incorporated
into the code.
%
A coarse ``goodness of fit'' estimator are the uncertainties obtained for the
best fit model. For low mass clusters, the stochasticity involved in their
generation will impact as larger overall uncertainties assigned to its
parameters. This is particularly noticeable for the age, as seen
in~Fig.\ref{fig:errors_mass}.\\

\begin{figure}[t]
    \centering
    \includegraphics[width=5in]{figs/errors_mass.png}
    \caption{Distribution of $\log(age)$ errors versus (normalized) masses.}
    \label{fig:errors_mass}
\end{figure}

To address this point more directly we added a sentence to the end of
paragraph two in this section, and another in the second to last paragraph of
Sect. 4.


% 13
\setcounter{subsection}{12}
\subsection{Sect. 4, paragraph 5}
We could use $z/z_{\odot}$ internally in \texttt{ASteCA}, but there would be
nothing to gain from it (programmatically speaking). Since this is a simple
re-scaling, it can be easily performed after the matching process is done. It
wouldn't change the fact that converting to [Fe/H] involves a logarithm, and
thus a mathematical dependence in the uncertainties of the form:

\begin{equation}
\begin{split}
% \begin{aligned}
[F/H] & = \log(X) \,\,;\,\, X=z/z_{\odot} \\
\sigma_{[Fe/H]} & = \sigma_{X} / [X \times \ln(10)] \;\;;\:\;  \sigma_{X}
= \sigma_{z} \times z_{\odot}
\end{split}
% \end{aligned}
\end{equation}

Hence if we used $z/z_{\odot}$ for the distribution of uncertainties in Fig 3,
the result would look like Fig. 1 in the cover letter from Rev 1, only re-scaled.
It wouldn't resolve the error dependence that arises in [Fe/H] due to the
logarithm.
%
Errors are thus purposely displayed using [Fe/H] (rather than z) to make the
reader aware that [Fe/H] errors increase for low metallicities, for a
mathematical reason.


% 18
\setcounter{subsection}{17}
\subsection{Sect. 5.1, paragraph 9 (near bottom of page 9)}
\label{sec:18}
(Notice that this question is related to Sect. 5.1, not 5.2 as stated in the
referee's letter)\\

The referee said that the ``tidal radius of the cluster is typically
significantly larger than the radius at which the surface number density of the
cluster equals that of the field''.
%
The tidal radius is mathematically equivalent to the radius where the number
density of stars reaches the field density (``density radius''). This can be
seen by solving King's number density function~\citep[][Eq. 14]{king_1962} for
zero density (assuming the relation has been corrected for the field
density), where the solution is $r=r_t$.

We assume that what the referee meant is that, \emph{in practice}, the density
radius is often set to more conservative values than the tidal radius.\\
% If this is the case, we can prove that this has little impact on the estimated
% masses.\\

Using Eq. 18 from~\cite{king_1962}, we can obtain the number of
cluster members up to a given radius. We use a constant core radius and vary
the tidal radius to process different concentrations ($c{=r}_t/r_c$).
%
In Fig.~\ref{fig:frac_memb_lost} we show the fraction of total members lost, for
five different density radius defined as fractions of the tidal radius.
%
\begin{figure}[t]
    \centering
    \includegraphics[width=5in]{figs/figure_1.png}
    \caption{Fraction of cluster members that are lost using the density radius
    rather than the tidal radius, as a function of cluster concentration 
    (assuming a King density profile). We show several density radii values,
    defined to be smaller (i.e., underestimated) than the tidal radius by
    given fractions.}
    \label{fig:frac_memb_lost}
\end{figure}
%
We can see that even if the density radius is underestimated by 50\% 
($r=0.5 r_t$), the number of stars that are lost reaches $\sim30\%$ only for
very low concentrations ($c<1.5$), and drops rapidly for more concentrated
clusters.
%
In a cluster like NGC 419 for example, with $c=6.9$~\citep{Goudfrooij_2014}
underestimating the tidal radius by $\sim50\%$ would result in 
losing less than $10\%$ of its members.

Furthermore, according to the mass segregation effect, we know that low-mass
stars are preferentially displaced towards the outer parts of the cluster (and
eventually lost via evaporation). Hence, stars located between the
density radius and the tidal radius, are mainly low-mass stars.
%
% If we calculate the distribution of stars in a cluster of fixed
% total mass, according to a certain IMF, we can estimate what fraction of the
% total cluster mass these low-mass stars represent.
% %
% This is shown in Fig.\ref{fig:num_mass_lost}. We used a~\citet{Kroupa_2002}
% IMF to generate a $10^4\,M_{\odot}$ cluster composed of $\sim26400$
% stars distributed between $0.01-100\,M_{\odot}$. The plot shows the (normalized)
% number of stars ordered from the lowest ($0.01\,M_{\odot}$) to the highest mass
% ($100\,M_{\odot}$), versus their (normalized) cumulative sum of individual
% masses.
% %
% We can see that the 50\% of lowest-mass stars in the cluster ($\sim13200$),
% represent less than 10\% of its total mass.\\
%
% \begin{figure}[t]
%     \centering
%     \includegraphics[width=5.in]{figs/figure_2.png}
%     \caption{Distribution of stars in a $10^4\,M_{\odot}$ cluster according to
%     the~\citet{Kroupa_2002} IMF, ordered from lower masses to the left
%     ($0.01\,M_{\odot}$), to larger masses to the right ($100\,M_{\odot}$).
%     In the vertical axis the cumulative (normalized) sum of masses is shown.}
%     \label{fig:num_mass_lost}
% \end{figure}
%
% In a real cluster, the distribution of masses will of course not be as ordered
% as it is in Fig.\ref{fig:num_mass_lost} (in the sense: smaller to larger),
% meaning that some stars with larger masses will also be located between the
% density and the tidal radius.
%
These stars are mostly not detected due to the effects of limiting
magnitude and photometric incompleteness (both accounted for in the synthetic
CMD generation algorithm). Extending the density radius would thus
include an even smaller percentage of cluster stars than those predicted
above.
%
For smaller low-concentration clusters, where the loss of members is seen to
have a larger impact, the mass uncertainties are large enough to contain this
effect.\\

% Finally, given that the density radius is positioned where the radial density
% profile reaches the field density, using a larger value would include mostly
% field stars.
% If we used a radius larger than the density radius, we would in a first step
% include more stars in the cluster region CMD; this is true. But we would lose
% all these extra stars in the following step, when the cluster CMD is cleaned
% from field stars contamination  (prior to the best synthetic cluster match
% process). The result is that the number of estimated members remains almost
% unchanged, and thus the estimated mass will also remain unchanged.\\

% The only way that extending the density radius could result in a larger mass
% estimate, is if no field star cleaning was performed previous to the best
% synthetic cluster match. This is of course, not correct.\\

We have added a paragraph in Sect 5.1 explaining the mass estimation in more
depth, mentioning this effect. A footnote was also added to the second to
last paragraph of this same section, also related to this effect.




% 19
\setcounter{subsection}{18}
\subsection{Sect. 5.2.1}
\label{sec:19}
\textbf{1. On the M/L relation of the H03 and P12 articles}\\

We agree with the referee that the mass-age correlation is the reason for the
discrepancies between these two parameters in the H03 and P12 articles. This is
mentioned in the second to last paragraph of Sect 5.2.1: ``These mass
differences [between H03 and P12] are closely related to $\log(age/yr)$
differences.''

To make this point clearer, we have made the following changes to the section:\\

\noindent 1. Fig. 11 was expanded to show the age-mass correlation in different
mass regimes. It was moved to the beginning of this section, and is now Fig.
9\\
2. The second to last paragraph from this section was moved to the beginning,
and edited to reflect changes made to Fig. 11(9).\\
3. The old Fig. 10 was removed, as it became redundant after the changes made
to Fig. 11(9).\\

Other changes made to Sect 5.2.1 are mentioned in the following point.\\\\

\noindent \textbf{2. Issues related to crowding in the central regions of
massive clusters}\\

% ``even for well-exposed HST images, the completeness in the central
% regions of massive LMC clusters decreases to about 50\% for stars with
% brightnesses $\sim3$ mag below the MSTO (see, e.g., Milone et al. 2009, A\&A,
% 497, 755), so that even HST photometry of individual stars will miss some
% significant part of the total mass (or light) of the cluster.''\\

The referee is \emph{absolutely correct} regarding the issue of crowding in our
photometry. Even though we investigated the effects of complicated cluster
features (such as the double red clump, or the extended main sequence turn-off),
we failed to realize that the much simpler effect of crowding could account for
the mass discrepancies found.
%
Although \texttt{ASteCA} considers how other photometric effects alter the
generated synthetic CMDs (limiting magnitude, faint stars incompleteness,
photometric errors), it does not include the loss of stars in the central
regions of a massive cluster due to photometric crowding.
%
Since the mass estimation is based on the number of observed stars, losing a
large portion of those will very clearly bias our results. The \texttt{ASteCA}
code was originally developed to analyze open clusters, which is why crowding
was never really an issue (until now of course). The effect of crowding on
synthetic CMDs will be added to the code as soon as possible.\\

Our Washington photometry for the clusters NGC419, NGC1917, and NGC1751, show
total numbers of estimated cluster stars around 2300, 170, and 200 in their
cluster region CMDs.
%
The synthetic CMDs of large masses (those given in the literature) show for
these same clusters a number of members around 32700
($M{=}2.7{\times}10^5\,M_{\odot}$), 3900 ($M{=}8{\times}10^4\,M_{\odot}$), and
6800 ($M{=}7.2{\times}10^4\,M_{\odot}$), for NGC419, NGC1917, and NGC1751
respectively.
%
This means that, to account for the differences in the number of observed versus
synthetic cluster stars, photometric crowding in our Washington data would need
to be responsible for the loss of ${\sim}95\%$ of the stars present in the
observed cluster regions.
%
This percentage of lost stars in the cluster region couldn't be accounted for by
any of the photometric process included in \texttt{ASteCA}.\\

We requested Dr Goudfrooij for the NGC419 and NGC1751 data used
in~\cite{Goudfrooij_2014}, who very kindly provided it to us. The photometric
tables were generated from observations performed with the HST ACS/WFC
instrument, which has a much higher resolution than our Washington photometry 
(0.05 arcsex/pixel versus 0.274 arcsec/pixel). The cluster data was processed
with \texttt{ASteCA}, imposing a cut on $F555W{=}23$ mag to minimize the number
of stars lost due to incompleteness (as instructed by Dr Goudfrooij). The
fundamental parameters were fixed to those given in~\cite{Goudfrooij_2014},
except the mass which was left to vary between
$10^3 - 3{\times}10^5\,M_{\odot}$, and $10^3 - 10^5\,M_{\odot}$, for NGC419 and
NGC1751 respectively.
%
Fig.\ref{fig:NGC_fit} shows the result of this process. The mass estimated
for NGC419 using the ACS/WFC dataset is $M{\approx}2.5{\times}10^5\,M_{\odot}$,
very much in line with the average literature estimates of
$M{\approx}2.7{\times}10^5\,M_{\odot}$.
A similar result is found for NGC1751, for which a mass of
$M{\approx}5{\times}10^4\,M_{\odot}$ is estimated, very close to the average
Goudfrooij et al.~value of $M{\approx}6.25{\times}10^4\,M_{\odot}$.

Imposing a similar cut on our NGC419 Washington data on $T_1{=}22$ mag (again,
to minimize the loss of faint stars due to crowding) we find a mass estimate
close to $10^5\,M_{\odot}$. This value is three times larger than our original
estimate for this cluster ($2.8{\times}10^4\,M_{\odot}$), and around three times
smaller than the estimate by Goudfrooij.
While the mass difference is still somewhat large, this proves that reducing the
loss of stars due to crowding improves our mass estimate substantially.
Our mass validation study in Appendix A missed this effect, because the synthetic
MASSCLEAN clusters are not affected by crowding.\\

We must conclude that the referee was right to question our conclusions
regarding the mass estimates for larger clusters. The effect of photometric
crowding in our Washington photometry is clearly the responsible for the
systematic underestimation of masses, when compared to the integrated photometry
studies.\\

\begin{figure}[t]
    \centering
    \includegraphics[width=5in]{figs/NGC_cut_fit.png}
    \caption{\emph{Top left}: ACS/WFC CMD for NGC419 (data tables by Dr
    Goudfrooij), with limiting magnitude cut at $F555W=23$ mag. \emph{Top right}:
    best fit found by \texttt{ASteCA} for the mass (the rest of the parameters
    were fixed).
    \emph{Bottom}: idem, for NGC1751.}
    \label{fig:NGC_fit}
\end{figure}

To address the issues mentioned in this point, we have changed the
section in the following way:\\

\noindent 1. Sect 5.2.1 was substantially shortened and simplified. The
exploration of other possible effects (double red clump, eMSTO, etc.) were
removed, as they are no longer needed.\\
2. Table 4 (which showed the 5 clusters with the largest mass discrepancies) was
removed.\\
3. The last paragraph was removed as it no longer reflects the overall
conclusions of this section.\\
4. The stellar crowding in our photometry is mentioned as the effect responsible
for the systematic mass differences.\\

Furthermore, the following changes were made to the rest of the article:\\

\noindent 1. A sentence was added to the second to last paragraph of Sect 3.1
mentioning this effect.\\
2. The Conclusions were edited to reflect the above changes.\\
3. A sentence was added to the first paragraph of Appendix A, explaining that
MASSCLEAN clusters are not affected by crowding.\\
4. Appendix D was removed, as it became irrelevant now that the
reason for the mass discrepancies is clear.


% 22
\setcounter{subsection}{21}
\subsection{Sect. 6.2, paragraph 5}
Changed, as suggested by the referee. The sentences ``Both MCs accumulate most
clusters... The SMC has a larger proportion of...'' that followed, were removed
to accommodate the changes made in Sect 5.2.1.


%-------------------------------------------------------------------
\bibliographystyle{aa}
\bibliography{biblio} % your references Yourfile.bib

\end{document}